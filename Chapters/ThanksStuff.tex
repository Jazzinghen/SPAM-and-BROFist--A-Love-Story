\chapter*{Ringraziamenti e Roba Varia}
\label{chap:Thanks}

\emph{Venerdì 17 Settembre 2010, Ore 19:00. Aula Studio del Padiglione 3
dell'FBK}

Anche se questo è il primo capitolo (anzi, addirittura direi che è lo
zeresimo o il ``menounesimo'') sappiate che è l'ultimo. Perchè? Boh.
Probabilmente volevo godermelo, perchè non volevo essere troppo di fretta
nello scriverlo.

\'E stata una dura lotta. Sono quasi sette mesi che lavoro dietro a questo
progetto. Ne ho passate di strane. Ho visto cose che speravo non avrei
visto. Ho provato cose alle quali pensavo di essere immune. Ho imparato
molto, cosa che mi ha dimostrato di quanto ero scarso all'inizio. Prima
pensavo di saper programmare. In realtà sapevo solo... Fare dello
\foreignword{scripting}. Adesso, grazie a questo progetto e anche grazie
all'esame \techname{Architettura degli Elaboratori} del mio relatore, il
professor \emph{Luigi Palopoli}, le mie viste si sono espanse. Come lo sono
le mie competenze tecniche.

Questo, ovviamente, non è stato solo grazie a me. Se non fosse stato per la
presenza e per l'aiuto datomi da certe persone sarei ancora là a bazzicare
tra le mie incertezze e le mie ``competenze'' incompetenti.

In questo tempo, inoltre, ho passato dei momenti assurdi. Se devo essere
sincero, per quanto mi siano parsi negativi, sono stati degli ottimi
momenti. Dovete capire che per me ogni momento che lascia qualcosa dentro è
un momento positivo. 

Ho passato una settimana dove pensavo di morire. Realmente. Andavo a letto
e tutto quello che riuscivo a pensare era che, probabilmente, il mio cuore
si sarebbe fermato durante la notte. Il giorno dopo, invece, avevo paura
di non riuscire ad arrivare a questo momento.

Durante un altro periodo dormivo quattro ore a notte. L'insonnia, sapete.
Dopo qualche giorno che si soffre d'insonnia le giornate sono accompagnate
da dei rari istanti dove la mente vaga e si perdono dei minuti dove,
magari, la gente ti ha detto qualcosa e non hai sentito. Oppure ancora
arrivano delle momentanee allucinazioni dove ti sembra di finire da un
piano all'altro, andare a sbattere contro le scale presenti dall'altra
parte della rampa che stai percorrendo, per poi accorgerti che sei fermo.

Tutti questi momenti, però, mi hanno dimostrato una cosa. Sono tutte
stupidaggini. Il nervosismo. Meh. Infatti eccomi quì. Ciò che non uccide
rafforza, giusto? Inoltre tutto questo mi ha dimostrato che per me è
possibile farlo. Arrivare in fondo ad un impegno così importante. Ho
passato un anno dove non ero assolutamente convinto, dove non vedevo la
fine.

\'E stato alla fine di quell'anno, dopo un periodo orribile, dove ho
camminato solo la sera di Natale da Ravina a Trento, che ho visto come
andava affrontata la vita, almeno la mia\footnote{Anche perchè non sono quì
per scrivere un libro da motivatore fallito. Se ne volete uno potete andare
in libreria. Ci sono già troppi motivatori con la loro filosofia da
Americani che riempiono la nostra testa di cose inutili.}.

Ed è stato così che lì, nella mia mente, c'è un qualcosa che mi spinge
avanti. Un obiettivo che mi permette di affrontare gli esami che non
comprendo, un motore che, per citare qualcosa a me caro\footnote{E anche a
molti di quelli che mi hanno seguito in questa epica Quest}, mi permette di
``fare l'impossibile, vedere l'impossibile [...] toccare l'intoccabile'' e
di focalizzare la mia forza di volontà, per quanto facilmente distraibile.

Ed è così che sono arrivato quì. Grazie a questo e grazie a tutte quelle
persone che sono state con me fino in fondo. O che, semplicemnte, hanno
bazzicato dalle mie parti anche per poco tempo.

\section*{Ringraziamenti}
Intanto vorrei dire una cosa a tutti. Scusatemi. \'E stata una cosa dura e
la mia mente era preparata a questo solo fino ad un certo punto. All'inizio
l'impatto è stato quello che è stato. Quando è iniziata l'esate, però, ho
iniziato a notare una cosa. La mia soglia dell'attenzione ha incominciato a
calare vertiginosamente. Quando non lavoravo alla tesi lasciavo vagare la
mente, probabilmente ignorando alcuni dei vostri discorsi.

Inoltre la mia soglia di sopportabilità è calata. Penso di aver trattato
male della gente che non se lo meritava, di aver ignorato altri. Anche se,
penso, probabilmente con alcuni ho anche trovato in questa situazione la
forma mentis per comportarmi come dovevo.

Comunque chi è importante per me comunque è stata riposta nella mia mente
piena di milioni e milioni d'insulti e di nervosismi in una zona a contatto
limitato con essi, perciò sono salvi. Vi voglio bene, ragazzi. \'E sempre
dura dirlo, ma adesso sono abbastanza calmo per esserne convinto.

Vorrei comunque fare una lista di persone che meritano delle
menzioni\footnote{Oh, suvvia, se non siete presenti in questa lista non
vuol dire che vi odio.}

Inizierei col ringraziare due persone che, probabilmente, stanno leggendo
questo testo e si stanno chiedendo \emph{``Ma perchè non può semplicemente
evitare?''}. Probabilmente pensavano anche che mi sarei dimenticato di
loro, ma come potrei? Anche se sono spesso fuori sono sempre là. E non
pensiate che non mi senta in colpa per il tempo extra che ci ho messo.
Grazie dunque ad Enzo ed a Luisa, i quali mi sopportano ogni giorno
dell'anno.

Un altro ringraziamento, ovviamente, va al professor Luigi Palopoli, il
quale si è trovato una persona che arriva in ritardo, che si dimentica
degli appuntamenti e che gli presenta dei prototipi in \techname{Python}.

Ed ora passiamo ad una lista che, spero, non diventerà troppo lunga.

Intanto grazie al gruppo che ha permesso, in pratica, che tutto questo
progetto arrivasse a conclusione, ovvero ``Il Team Dell'Aula Studio'', i
cui membri sono:
\begin{description}
    \item[Giovanni ``Il Giovane Dacav'' Simoni]guru della programmazione e
        Troll ultimativo
    \item[Ivan ``GODS'' Simonini]``Colui che piega la carta'', mastro
        artigiano di LEGO e \foreignword{Rage Developer}
    \item[Gabriele ``Ranger'' Depedri]allenatore di Taiji, Santo e creatore
        di personaggi \emph{da BG}
    \item[Alessio ``Warriors'' Guerrieri]colui che ha visto l'algoritmo
        dietro gli algoritmi
\end{description}
Un sacco di altra gente gira per la sala studio, in realtà. Questo, però, è
il gruppo che è sempre da quelle parti. Ogni volta che arrivo trovo almeno
uno di loro, pronto al prossimo \emph{Coffee Time} o a fare della Scienza.

Devo ringraziare anche un altro gruppo di persone, ovvero gli altri due
membri, oltre al sottoscritto, del ``Circolo delle Ladies e dei Lords
Bevitori'', ovvero:
\begin{description}
    \item[Michele ``Master'' Graffeo]``colui che ci rovinerà tutti'',
        ricercatore dagli occhi iniettati di sangue, \emph{IL Master}
    \item[Diana ``Dia'' Fantappiè]l'ornitorinco dell'architettura,
        grandissima incasinatrice di computers
\end{description}
A loro grazie, sia per le serate spese a bere ed a parlare del più che del
meno, sia, a Michele, grazie per l'avermi sopportato, visto che non lo
lascio mai in pace e visto che mi ha mostrato in che direzione bisogna
andare.

Un grazie anche ai guerrieri che con i loro atti eroici (o le loro
\emph{trassate}) finiranno nel \techname{Valhalla} a lamentarsi del nuovo
regolamento di \techname{Warhammer 40K}! Sarebbe lunga elencarvi tutti,
però devo un grazie soprattutto ad \emph{Ivan Candiolli}, l'Odino della
squadra, per avermi dimostrato che aiutare il prossimo e non reagire
malamente porta al rispetto e non allo sfruttamento.

Come dimenticarsi, inoltre, dei due negozianti più strani che conosca,
oltre che ad amici (o più) di vecchia data? Grazie a Chiara Finessi ed Igor
Sontacchi. Con loro abbiamo creato, progettato, organizzato, imprecato per
i problemi emersi durante le manifestazioni, tutto. Oddio. Se penso che
domani e domenica devo andare a dare una mano muoio, però s'ha da fare. Ci
vediamo in settimana, ragazzi, tanto so dove trovarvi.

Da queste parti dovrebbero esserci anche Luca ``Luber'' Weber e Carlo
``Farlo''\footnote{Ma anche Corlo, Forlo, Orko \dots{}} Lorenzetti. Gente
che si è bevuta il cervello, in una maniera o nell'altra, che soffre per le
decisioni di una casa di produzione di giochi di miniature (capisco la
vostra sofferenza, ragazzi), giocatori di videogames come me.

Penso di essere arrivato alla fine. Potevo ringraziare un sacco di altra
gente, ma non posso scrievere più quì che in tutta la tesi, no? Ancora due
sezioncine e poi cominciamo.

Ah! Grazie anche alle bariste, senza di voi non so se sarei arrivato in
fondo alle peggiori giornate che ho visto durante questo tempo. Verrò a
prendere un caffè lunedì mattina, al solito.

\section*{Musica}
Durante questo periodo ho ascoltato moltissima musica, però solo alcune
canzoni hanno dato una vera spinta al progetto. Vorrei redarre una veloce
lista per tutti quelli che fossero interessati a cosa ascoltavo negli
ultimi mesi.
\begin{description}
    \item[Snake Eater]- \emph{Cynthia Harrel}
    \item[Voyage to Avalon (Orchestra Edition)] - \emph{Kenji Kawaii}
    \item[If you want blood (You've got it)] - \emph{AC/DC}
    \item[Wolves of the sea] - \emph{Alestorm}
    \item[Through the fire and the flames] - \emph{DragonForce}
    \item[Drunken Lullabies]- \emph{Flogging Molly}
    \item[Inner Universe] - \emph{Yoko Kanno/Origa/Shanti Snyder}
    \item[``Libera me'' From Hell] - \emph{Tarantula/Yuri Kasahara}
    \item[Halo Theme (Mjolnir Mix)] - \emph{O'Donnell/Salvatori}
    \item[Invaders Must Die] - \emph{Prodigy}
    \item[Lady Jam] - \emph{E-Z Rollers}
    \item[Don't Be Afraid] - \emph{Elisa Fiorillo}
\end{description}

\section*{SPAM: Origine di un nome}
\image  {Pictures/SPAM}
        {Un'immagine dello \SPAM{} nella sua ultima versione,
            sviluppata dal Maestro Ivan ``GODS'' Simonini.}
        {.8}
        {img:spam}
In giro per il testo potrei aver utilizzato il nome \SPAM{}. Devo ammettere
che questa scelta potrebbe creare della confusione nei lettori. Ma non vi
preoccupate, ora spiegher\'o in maniera abbastanza rapida cosa significa.

All'inizio l'unità \techname{NXT} che utilizzavo non aveva un nome, perciò
per riferirmi ad essa utilizzavo sempre termini come ``Quella là'', oppure
``Il robottino della mia Tesi''. Era ora di trovare un nome che la gente
ricordasse e potesse ricondurre automaticamente al'hardware che ho
utilizzato per il progetto.

Dopo svariate opzioni sono arrivato, all'inizio, al nome \SPAM{}, il quale
significava, all'inizio \techname{Super-Powerful Automated Mecha}. \'E
anche vero che avevo scelto il nome come tributo ad un gruppo comico molto
conosciuto che mi ha fatto ridere più e più volte col loro humor inglese,
il quale mi ha tradito solo poche volte\footnote{Vedi \emph{Il senso della
vita}}. I \emph{Monty Python}, in effetti, hanno fatto in modo che il mondo
utilizzasse la parola \SPAM{} per le eMail e le pubblicità che intasano le
caselle di tutto il mondo.

\'E stato così che ho deciso di utilizzare \SPAM{} come soprannome per il
robot base per il mio progetto. Il significato, però, dopo i primi due
mesi, è cambiato, a causa soprattutto delle prestazioni dei motori i quali,
anche se non ridotti, sono estremamente lenti.

Alla fine ho deciso per cambiare significato all'acronimo, mantenendo
quello vecchio, in \techname{Semi-Powerful Automated Mecha}, molto più
appropriato per le potenzialità fisiche del robot.

\section*{Gli strumenti}
Questa tesi è stata scritta su \techname{Jehuty}, un \emph{Sony VAIO
VPCEB1A4}, utilizzando vari software:
\begin{itemize}
    \item VImproved (VIM per gli amici)
    \item Gedit
    \item SciCosLab
\end{itemize}

\cleardoublepage

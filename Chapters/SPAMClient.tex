\chapter{Il Client di BROFist per SPAM}
\label{chap:SPAMC}

In questo capitolo parleremo dello svilupppo del client per \SPAM{} del
progetto \BROFist{}, dei problemi riscontrati e delle funzionalità che
possono venir riutilizzate in altri progetti.

\section{Creazione di SPAM}
Il robot sul quale è stato creato l'intero progetto non è sempre stato
nella forma attuale. All'inizio il modello utilizzato era quello proposto
dalla \emph{Lego} nelle istruzioni base contenute nelle scatole dei
\nxt{} (Si faccia riferimento all'immagine \ref{img:orignxt}).
\image  {Pictures/StandardNXT}
        {Una delle versioni base di un \nxt{}, costruita seguendo le
        istruzioni contenute nei kit forniti dalla lego}
        {.5}
        {img:orignxt}

\graffito{Il problema di usare il modello presentato dalla \emph{Lego}} 
Dopo poco, però, anche grazie all'aiuto di Ivan Simonini, si è capito che
quella struttura aumentava gli errori di rotazione dei motori a causa di come era
stato pensato.

I motori erano montati in maniera tale che tutto il peso dell'unità
gravasse a metà tra i due, dove non era presente nessun punto di scarico
delle forze. Questo portava ad un'inclinazione verso il centro delle ruote,
facendo sì che non si muovessero in linea retta, richiedendo l'uso di un
feedback controller.

Viste le premesse è stato richiesto ad Ivan Simonini, che al tempo lavorava
come esperto presso il museo di scienze naturali di Trento, di controllare
l'unità e di renderla più strutturalmente stabile.

Il processo ha richiesto un mese e due prototipi, anche a causa della
scarsità di parti presenti nelle scatole \emph{Educational} dei \nxt{} e
delle richieste iniziali per il progetto, ovvero avere la possibilità di
montare un sensore di distanza brandeggiabile parallelo al terreno ed un
sensore di luce puntato verso il terreno a circa un centimetro da esso.
\graffito{La necessità di avere il sensore parallelo al terreno}
Il sensore di distanza doveva venir necessariamente montato in maniera
parallela rispetto al terreno in quanto il sistema di rilevamento della
distanza funziona con l'utilizzo di ultrasuoni. Questi vengono emessi dal
sensore il quale, misurando dopo quanto tempo ritorna l'onda, calcola la
distanza dall'ostacolo puntato (Fare riferimento all'immagine
\ref{img:sonarpr}).
\image  {Images/SonarPrinciple}
        {Teoria base del Sonar Attivo, utilizzata dai sensore di distanza
        ad ultrasuoni per \nxt{}}
        {1}
        {img:sonarpr}

Questo sistema, però, è influenzato dall'angolazione della superficie
rispetto all'origine dell'onda, in quanto, se questa non è perpendicolare o
di pochi gradi differente rispetto all'angolo retto, l'onda 
rimbalza in una direzione differente da quella d'origine. Questo fa in
modo che non torni al sensore, il quale pensa di non aver alcun ostacolo di
fronte a se, oppure di averlo più lontano di quanto non lo sia realmente.

Proprio a causa di questo problema si fa necessario il montaggio dei
sensori in posizione parallela rispetto al terreno per eliminare almeno un
fattore di errore per la lettura della distanza.

Alla fine del processo di sviluppo e creazione, comunque, è stata costruita
l'unità \SPAM{} (Foto \ref{img:spam}), la quale ha un sensore di distanza
brandeggiabile parallelo al terreno, un sensore di luce e l'accesso diretto
al brick \emph{NXT}.

\section{Struttura e funzionamento del Client}\label{sec:clientstru}
Come già detto in precedenza (sezione \ref{sec:BROClient}) il client
\BROFist{} gira su un sistema \nxtOSEK{}, il quale fornisce tutte le
funzionalità real time necessarie all'esecuzione con deadlines richieste da
questo tipo di progetto. Il client è formato da quattro Task, i quali
collaborano per il controllo dello \SPAM{} e per la comunicazione con il PC.

Per ricapitolare i quattro Task hanno le seguenti mansioni:
\begin{itemize}
    \item Comunicazioni Bluetooth ed interpretazione dei
        BROFists (\ref{sec:BROCommS})
    \item Calcolo delle velocità istantanee dei motori (\ref{sec:BROSpdS})
    \item Calcolo ed attuazione delle potenze dei singoli
        motori (\ref{sec:BROPIDS})
    \item Output di dati sul Display LCD (\ref{sec:BROLCDS})
\end{itemize}

Vediamo rapidamente in dettaglio ogni Task, con alcuni dati tecnici, per
avere un'idea più chiara di quello che fanno:

\subsection[Task ``BRO\_Comm'']{Task ``BRO\_Comm'': Comunicazioni BT ed
interpretazione dei BROFists}\label{sec:BROCommS} 
Questo task è, fondamentalmente, il cuore di tutto il client. \'E
incaricato di comunicare via BlueTooth con il server, ricevere i dati,
interpretarli, far fare allo \SPAM{} ciò che è stato richiesto e quindi
mandare un'eventuale risposta al server.

Per fare questo è stato necessario un lungo periodo di ricerche e test per
comprendere come funzionino le API per la gestione delle comunicazioni
BlueTooth ed avere ben in mente come fosse strutturato il protocollo di
comunicazione \BROFist{}, cose delle quali parleremo nelle sezioni
\ref{sec:BROTooth} e \ref{sec:theProtocol}.

Dopo aver superato gli ostacoli imposti dai device drivers (e dal
controller BlueTooth in se) lo svilupppo ha portato alla scrittura di un
Task con un'esecuzione ciclica ogni \cypher{2ms}, il quale recupera sette
\emph{BROFists} contemporaneamente quando sono presenti all'interno del
buffer del controller BlueTooth e quindi li interpreta.

\graffito{Perchè sette \emph{BROFists}}
La scelta di inviare sette \emph{BROFists} alla volta è data dal fatto che
sono presenti solo sette porte per la connessione di motori o sensori,
quindi l'idea è che le richieste non dovrebbero essere superiori a
quel numero. \'E anche vero che le persone potrebbero richiedere due dati
differenti dallo stesso Servo Motore, però si è considerato che un
programmatore possa derivare uno dei due valori avendo il primo.

\graffito{L'uso dei dati ricevuti}
Quando le richieste riguardano l'invio del dato letto da un sensore non ci
sono problemi nella gestione dei dati: i dati vengono letti dai sensori e
quindi vengono inseriti all'interno della struttura che andrà inviata al
server presente sul PC. Se viene inviato un valore di potenza o velocità
per i motori, invece, viene o chiamata la funzione per il calcolo della
potenza in base alla velocità oppure settato il primo valore all'interno
dell'array contenenti le potenze dei motori\footnote{Si faccia riferimento
alla sezione \ref{sec:BROStruct} per avere più informazioni riguardo alla
conformazione della struttura dei dati per il controllo dei motori}.

\graffito{I dati in uscita}
Una volta interpretati tutti i \emph{BROFists} il client forma una serie di
\emph{BROFists} di risposta, i quali contengono la richiesta ed il dato ad
essa correlato. Nel caso di potenza o velocità dei motori il dato è sempre
$0$.

\subsection[Task ``Speed\_Updater'']{Task ``Speed\_Updater'': Calcolo delle
velocità istantanee}\label{sec:BROSpdS}
La possibilità, da parte del server, di richiedere la velocità media di uno
dei servo motori e la presenza di un \PID{} ha reso necessaria l'aggiunta
di un Task per il monitoraggio e per l'aggiornamento dei dati riguardanti i
vari Servo Motori installati.

Il concetto di questo Task è molto semplice: ogni \cypher{5ms} viene letto i valori
dei Tachiometri presenti all'interno di ogni motore e vengono registrati
all'interno di un array circolare. Usando il valore precedente viene
calcolata la velocità istantanea di ogni valore e viene inserita
all'interno di un secondo array circolare.

Facendo ciò, dunque, avremo abbastanza dati per poter calcolare la velocità
media su \cypher{(5 * Dimensione dell'array circolare)ms}.

\subsubsection{Problemi derivanti dall'uso dei valori dei Tachiometri}
Il valore di ogni Tachiometro viene registrato all'interno di una variabile
di tipo \codeconst{int32_t}. Questo \emph{potrebbe} portare a dei problemi
dati dall'\emph{overflow} o dall'\emph{underflow} dato dall'eccessivo
utilizzo dei motori.

Questo, però, occorrerebbe soltanto dopo una rotazione continua, in una o
in un'altra direzione, di \cypher{$2^{31}$}°, ovvero
\cypher{2,147,483,648}°. Considerando il diametro delle ruote installate
sullo \SPAM{}, che sono quelle contenute nei Kit \techname{Educational} dei
\nxt{}, che misura \cypher{56mm} ed il fatto che esse sono collegate ai
Servo Motori tramite degli ingranaggi con un rapporto \cypher{1:1}, la
distanza che andrebbe percorsa da un singolo motore dovrebbe essere: $56mm
* 3.14 * (2,147,483,648 / 360)$, ovvero una distanza pari a $1,048,927m$.

\graffito{Un problema che non è un problema}
Questa distanza è fin troppo grande per venir considerata un problema,
visto che la carica della batteria e la velocità dei motori non
permetterebbero il raggiungimento di una cifra così grande.

\'E facile, comunque, che una frase simile si sia già sentita in passato nella
storia dell'informatica, portando a problemi e bug, ma la correzione di un
problema simile è facile e necessario solo in progetti che prevedono
l'utilizzo dei motori per una durata così lunga, perciò è stata lasciata in
dietro quasi volontariamente.

\subsubsection{Utilizzo di un array circolare per il calcolo della velocità}
Purtroppo la sensibilità dei Tachiometri integrati nei servo motori dei
\nxt{} è limitata ai gradi interi. Questo porta ad una propagazione
dell'errore di lettura quando si deve calcolare la velocità in
\cypher{Gradi al Secondo}, in quanto bisogna moltiplicare il valore
ricavato per una scala pari a \cypher{100 / Tempo di Sampling}.

A causa di questo, infatti, la sensibilità della velocità diventa, appunto,
pari alla scala che va applicata alla velocità per ogni intervallo di
Sampling.

Una sensibilità tale non è utile, in quanto inusabile dai vari algoritmi
di controllo\footnote{Anche perchè, in questa maniera, sarebbe obbligatorio
utilizzare come velocità target dei valori che sono dei multipli della
scala} visto che fanno uso dell'errore di velocità per ricavare la potenza da
applicare ai motori. Bisogna perciò trovare un altro metodo per il calcolo della
velocità media di un motore con i dati che ricaviamo dalla lettura
periodica dei Tachiometri.

\'E per questo che si è deciso di utilizzare una media mobile che usa le
ultime $10$ velocità istantanee.

\subsection[Task ``PID\_Controller'']{Task ``PID\_Controller'': Calcolo ed
attuazione delle potenze dei Motori}\label{sec:BROPIDS}
Siccome, anche a causa delle ``scarse''\footnote{Si faccia riferimento alla
sezione \ref{sec:BROTooth} per avere un'idea di questo problema}
prestazioni del BlueTooth sotto \nxtOSEK{}, non sappiamo esattamente quando 
saranno presenti dei nuovi dati per il controllo dei motori abbiamo bisogno
di un Task per il controllo periodico dei motori.

\'E per questo motivo che è stato implementato un Task periodico che, ogni
$5ms$, utilizza i dati per regolare la potenza dei vari motori.

Nel caso di controllo diretto delle potenze dei motori questo compito non è
particolarmente grave. Al Task, infatti, basta utilizzare l'ultimo valore
ricevuto come potenza da settare per il motore interessato.

\graffito{L'uso del \PID{}}
Quando, invece, viene richiesto di settare una determinata velocità il Task
deve far uso del \PID{}\footnote{Si faccia riferimento alla sezione
\ref{sec:PIDTheo} per un'idea migliore di cosa sia un \PID{}} implementato
tra le API del progetto.

Questo usa le ultime tre potenze utilizzate per il motore preso in
considerazione e le ultime tre differenze tra le velocità medie e la
velocità richiesta per calcolare la potenza da utilizzare per fare in modo
che i motori vadano alla velocità corretta. Una volta calcolato questo
valore lo utilizzano per far muovere i motori alla velocità richiesta
dall'utente.

\subsection[Task ``DisplayTask'']{Task ``DisplayTask'': Visualizzazione dei
dati sullo schermo LCD}\label{sec:BROLCDS}
Il compito di questo Task, eseguito ogni $1000ms$, è quello di fornire un
contatto con l'utente attraverso l'output di vari dati sull'LCD del Brick
dello \SPAM{}.

Questi dati sono configurabili, ovviamente, ma nella versione standard del
Client \BROFist{} i dati mostrati sono quelli raccolti dalla chiamata di
\nxtOSEK{} \codeconst{ecrobot\_status\_monitor(target\_name)}.

\subsection{Il Protocollo per \BROFist{}}\label{sec:theProtocol}
Per la comunicazione tra \SPAM{} e PC, ovviamente, si è resi necessari lo
studio e la realizzazione di un protocollo con il quale codificare le
richieste per lo \SPAM{}.

Per rendere questo lavoro più semplice è stato optato l'approccio più
facilmente implementabile in C, ovvero l'utilizzo di una struttura
contenente:
\graffito{Struttura di un \BROFist{}}
\begin{itemize}
    \item La codifica di quale sensore richiedere o di quale tipo di valore
        per il controllo dei motori è stato inviato. Per questa codifica si
        è deciso di utilizzare una variabile di tipo \codeconst{uint8_t},
        in quanto si è considerato che non esisteranno più di
        254\footnote{In quanto uno dei comandi dev'essere quello che
        dichiara finita la comunicazione} comandi
        differenti\footnote{Comunque la struttura del progetto è tale che,
        quando qualcuno deciderà di implementare più del massimo consentito
        da questo tipo di variabile, questo cambiamento sarà rapido e senza
        troppi problemi}
    \item La porta alla quale il sensore o servo motore è collegato. Anche
        questa codifica viene registrata all'interno di una variabile di
        tipo \codeconst{uint8_t}, vista la presenza di solo $7$ porte sul
        Brick \emph{NXT}.
    \item Il valore da settare in caso venisse inviata una richiesta di
        controllo di servo motori oppure il valore letto dal sensore
        richiesto. Per questo valore viene usata una variabile di tipo
        \codeconst{float}, in quanto la velocità richiesta potrebbe essere
        di tipo decimale, come la velocità media di un servo motore.
\end{itemize}

\graffito{Le codifiche delle richieste}
Al momento sono stati utilizzate le seguenti codifiche per le
richieste\footnote{Le quali sono definite, assieme a quelle per le porte,
all'interno del file header \cypher{broi\_fist.h} posto all'interno della
cartella \cypher{src/headers}}:
\begin{description}
    \item[LIGHT\_SENSOR] codificato come $1$, serve per la richiesta del
        valore letto da un sensore di luce
    \item[TOUCH\_SENSOR] codificato come $2$, serve per la richiesta del
        valore di un sensore di tocco
    \item[SOUND\_SENSOR] codificato come $3$, serve per la richiesta del
        valore registrato da un sensore di suono
    \item[RADAR\_SENSOR] codificato come $4$, serve per la richiesta della
        distanza letta da un sensore di distanza ad ultrasuoni
    \item[TACHO\_COUNT] codificato come $6$, serve per la richiesta del
        valore attuale del Tachimetro di un motore
    \item[AVG\_SPEED] codificato come $7$, serve per la richiesta della
        velocità media di un motore
    \item[SET\_SPEED] codificato come $8$, serve per inviare una velocità da
        settare ad un motore
    \item[SET\_POWER] codificato come $9$, serve per inviare una potenza
        ``grezza'' da settare ad un motore
\end{description}

Mentre le codifiche delle porte sono le seguenti:
\begin{description}
    \item[PORT\_1\textasciitilde{}4] codificate con i valori
        $1$\textasciitilde{}$4$, servono per applicare la richiesta ad una
        delle quattro porte per i sensori.
    \item[MOTOR\_A\textasciitilde{}C] codificate con i valori
        $5$\textasciitilde{}$7$, servono per applicare la richiesta ad una
        delle tre porte per i motori.
\end{description}

\section{BlueTooth e relativi problemi sullo SPAM}
I \nxt{} hanno due modi per comunicare con l'esterno: usando
l'\techname{USB} oppure il \techname{BlueTooth}. Nel progetto \BROFist{} si
è deciso di utilizzare la tecnologia \techname{BlueTooth} visto che il
progetto era puntato ad un utilizzo per la rilevazione telemetrica e per il
controllo realtime di un'unità \emph{NXT}, cosa che poteva risultare
complicata con la presenza costante di un cavo USB collegato all'unità.

Parleremo dunque prima della tecnologia in generale per poi spostarci ai
vari test che sono stati effettuati per calcolare la latenza delle
comunicazioni tra PC e \SPAM{} ed, infine, parleremo dei problemi
riscontrati nell'uso delle API di \nxtOSEK{} per l'uso del
\techname{BlueTooth}.

\subsection{Il BlueTooth}
Il \techname{BlueTooth} è uno standard Open per le comunicazioni a corto
raggio sviluppato dalla \emph{Ericsson} nel 1994. Questo permette la creazione di
\techname{Personal Area Network} ad alti livelli di sicurezza.

\graffito{Origine del nome BlueTooth}
Questa tecnologia deriva il suo nome dalla traduzione in inglese
dell'epiteto dato al re ``Aroldo I di Danimarca''. Il logo, invece, è la
runa derivata dall'unione delle rune che sono le iniziative del nome di
Aroldo: \textarn{h} e \textarn{b}.

\graffito{Funzionamento del BlueTooth}
Il \techname{Bluetooth} funziona tramite l'utilizzo di una tecnologia radio
chiamata \emph{Frequency-Hopping Spread Spectrum}. Questa tecnologia
suddivide i dati che vanno inviati e li trasmette su un numero massimo di
79 bande da $1$MHz l'una, tutte comprese nel range da $2402$ a $2480$MHz.

Il \techname{BlueTooh} utilizza un protocollo di comunicazione a pacchetti
con una struttura \emph{Master-Slave}, la quale permette la presenza di un
\emph{Master} e fino a $7$ \emph{Slaves} all'interno della stessa
\emph{picoNet}. Lo scambio di pacchetti viene regolato sul clock base,
regolato dal Master, il quale ha un intervallo di $312.5\mu{}s$. Il Master
utilizza gli slot per la comunicazione in maniera alternata: invia negli
slot pari e riceve in quelli dispari. I vari client, al contrario, ricevono
durante gli slot pari ed inviano in quelli dispari.

Questa struttura permette una comunicazione a corto raggio con un basso
consumo energetico. Il raggio di comunicazione è dato dalla classe del
device Bluetooth, andando da $1$m fino a $100$m.

La velocità massima di trasferimento è, nella maggior parte dei casi, di
$24$Mbit/s.

\subsubsection{La latenza della comunicazione BlueTooth tra PC ed
NXT}\label{sec:BROTooth} 
Prima di sviluppare completamente il sistema bisognava essere sicuri di
quanto tempo trascorresse tra l'invio di un \BROFist{} e la ricezione della
risposta, oltre al numero di errori di trasmissioni.

Per questo sono stati fatti un gran numero di test, prima solo in
ricezione, poi sia in invio che in ricezione. Questi test prevedevano il
cambiare distanza tra PC e \SPAM{}, aumentare gradualmente il numero di
test oppure cambiare la dimensione dei pacchetti inviati.

\graffito{I test di comunicazione SPAM $\rightarrow$ PC}
La prima serie di test effettuati prevedeva la ricezione di pacchetti in
serie spediti direttamente dallo \SPAM{} senza invii da parte del PC per la
sincronizzazione. Dopo una serie di test si è notato che la velocità di
comunicazione si attestava intorno ai $2ms$ e che non c'erano perdite di
dati durante le comunicazioni.

\graffito{I test di comunicazione PC $\rightarrow$ SPAM $\rightarrow$ PC}
Una volta completati i test per la comunicazione unidirezionale è stato il
turno di quelli bidirezionali, i quali prevedevano una sincronizzazione tra
PC e \SPAM{}.

Durante la prima fase dei test, eseguiti a varie distanze, tutte tra $1$ e
$10$ metri, la latenza delle comunicazioni, eseguite $100000$ alla volta, si
attestava intorno ai $4ms$.

Una volta controllati gli errori di comunicazione, però, si è notato che i
pacchetti ricevuti erano tutti errati. Molti dati non arrivavano,
sostituiti da valori casuali, tipici di letture di settori di memoria
errati. Questo ha portato ad una serie di test per la comprensione di quale
fosse il problema, oltre che a delle lunghe sessioni di \emph{Bugfixing}.
Dopo quasi un'intera settimana di lavoro intensivo, più di $50$ versioni
differenti di Server e Client e del \emph{Retro-Engineering} del codice dei
Device drivers di \nxtOSEK{} si è finalmente capito quale fosse il
problema.

\graffito{Problemi del BlueTooth sotto \nxtOSEK{}}
Il problema risiede sia nei Device Drivers che nel Controller BlueTooth
dei \nxt{}. La parte di problema dei Device Driver parte dal fatto che, pur
essendo presente un \techname{DMA} sulla piattaforma Hardware, non vengano
utilizzati gli Interrupt per gestire le comunicazioni via BlueTooth.
Infatti la chiamata che legge i dati da BlueTooth altro non fa che
controllare se sono presenti dei dati nel Buffer del
Controller\footnote{Richiedendo queste informazioni niente di meno che al
\techname{DMA} stesso} e quindi copiarli solo se sono esattamente quanti
indicati nel primo byte di dati ricevuti\footnote{In realtà lo spazio
``riservato'' per questa informazione dovrebbe essere di 2 byte, il secondo
dei quali viene ignorato dalla chiamata. Inoltre bisogna passare, come
parametro della funzione di lettura dei dati, la quantità di dati da
leggere. Una ridondanza di dati quasi inutile.} Questo fa sì che sia
necessario l'utilizzo di un approccio di \emph{polling} non bloccante,
creando un Task ad esecuzione periodica.

Questo Task dovrà agire soltanto quando ci sarà una risposta affermativa da
parte della funzione di lettura dei dati. Il problema dato da questo
approccio è quello derivato dalla perdita di tempo in caso i dati si
rendano disponibili tra un'esecuzione del Task ed un'altra.

Il ritardo, però, è stato limitato dando al Task un periodo di $2ms$,
facendo in modo che non possa trascorrere più di $1ms$ tra la ricezione
completa e l'utilizzo dei dati.

Utilizzando questo metodo, dopo una lunga serie di test da $30000$
comunicazioni bidirezionali alla volta, ha portato la latenza a $45ms$, ma
il numero di errori è stato portato a $0$.

Questo, dunque, ci porta ad un'altra considerazione: se il ritardo è al
massimo di $1ms$ vuol dire che l'errore è in un altro punto. Questo ci
porta all'idea che, probabilmente, le prestazioni del Controller BlueTooth
installato sui \nxt{} non siano ottime come si sperava. Questa, però, è una
congettura in quanto non c'è un modo sicuro per dimostrare questo fatto.

\section{Problemi di sviluppo}
Durante lo sviluppo del Client sullo \SPAM{} e anche durante il
conseguimento delle abilità necessarie per il suo sviluppo sotto
\nxtOSEK{} ci sono stati dei problemi che hanno portato ad un sensibile
rallentamento di queste attività.

Oltre al problema di dover creare l'immagine da caricare sull'\nxt{} ogni
volta e poi caricarla riavviando tutto il sistema, cosa che prendeva al
mimimo un minuto al ciclo di spegnimento, carica dell'immagine ed
esecuzione, l'impedimento più grande è stato quello della mancanza di un
debugger.

Ogni volta che si presentava un bug non c'era modo di capire da cosa fosse
dovuto se non con una lunghissima serie di test e di letture di dati sullo
schermo LCD. Inoltre neanche quest'ultimo metodo era sempre disponibile in
quanto quando avveniva un qualche problema di lettura dei dati dalla
memoria sullo schermo compariva il messaggio \cypher{DATA ABORT} e l'intera
unità si bloccava, rimuovendo qualunque messaggio che era stato scritto
sul Display.

\cleardoublepage

\chapter{Utilizzare i blocchi del progetto BROFist}
I blocchi sono stati configurati per essere il più comodi possibili per
l'utente senza che questo debba cambiare il codice di ognuno di essi ogni
volta che richiede una delle funzionalità progettate.

Siccome SciCos non permette una buona interazione con l'utente da parte del
programmatore, però, alcune delle funzionalità potrebbero risultare
complicate da comprendere, perciò verrà redatto un breve \emph{tutorial}
per l'utilizzo dei blocchi per \SciCos{} del progetto \BROFist{}.

\section{Preparazione dell'ambiente di sviluppo}\label{sec:SciComp}
Prima dell'utilizzo dei blocchi bisogna preparare l'ambiente per l'utilizzo
dei blocchi.

Ogni volta che si vule collegare una funzione scritta in \techname{C} a
\SciCosLab{} è necessario prima farla compilare al software, il quale
creerà delle librerie create apposta per la loro interazione con
\SciCosLab{}. Per fare ciò bisogna chiamare il comando
\cypher{ilib\_for\_link}, il quale compila queste librerie, quindi
collegare in \SciCosLab{} queste librerie alle funzioni volute con
\cypher{link}.

In questa maniera inseriremo le funzioni all'interno dell'ambiente, facendo
in modo che \SciCosLab{} possa collegare determinate chiamate alle funzioni
contenute nelle librerie.

Il passo successivo è quello di aggiungere all'ambiente le definizioni dei
blocchi di controllo sviluppati appositamente per queste funzioni (In
quanto così potrebbero venir chiamate solo da \techname{SciLab}). Per fare
questo bisogna caricare i file contenenti le definizioni con il comando
\cypher{exec}.

Una volta fatto tutto il procedimento si potrà procedere alla
modellizzazione grafica, spiegata sotto.

\graffito{Lo script di caricamento}
Per facilitare questo procedimento è stato aggiunto un file di script per
\SciCosLab{}, il quale fa tutte queste operazioni in ordine. Per eseguire
questo file di scripting basta scrivere nella finestra di \techname{SciLab}
\cypher{exec BRO\_loader.sce}. Questo compilerà, se necessario, tutte le
librerie, farà il \foreignword{linking} delle funzioni ad esse associate e
quindi caricherà le definizioni di tutti i blocchi scritti apposta per il
progetto \BROFist{}.

\section{Utilizzare i blocchi per SciCos}
Una volta fatto il caricamento ed entrati all'interno del modellatore
grafico bisogna caricare i blocchi e collegarli in maniera corretta per il
loro utilizzo.

Per caricare un blocco del progetto basta, dentro \SciCos{}, andare sotto
\emph{Edit $\rightarrow{}$ Add new block} e quindi inserire il nome del
blocco che si vuole aggiungere al modello.

Come prima cosa andrebbe caricato il blocco \emph{BRO\_Controller}, il
quale andrà collegato a tutti i vari blocchi che effettuano le connessioni
(Immagine \ref{img:Tut01}).

\image  {Images/Tutorial/Step01}
        {Inserimento del blocco \emph{BRO\_Controller}}
        {.8}
        {img:Tut01}

Una volta fatto questo bisognerà aggiungere un blocco per la comunicazione
con lo \SPAM{}. Questo blocco potrà essere o un \emph{SENS\_Disp} o un
\emph{MOTOR\_Disp}. Noi faremo un esempio utilizzando un
\foreignword{dispatcher} per i motori. Una volta posto il blocco
nell'editor dovremmo provvedere a collegare alla prima porta in input di
questo blocco la porta in uscita del \foreignword{controller} per le
comunicazioni (Immagine \ref{img:Tut02}).

\image  {Images/Tutorial/Step02}
        {Inserimento del blocco \emph{MOTOR\_Disp}}
        {.8}
        {img:Tut02}

Il prossimo passo sarà aggiungere, configurare e collegare i blocchi
per le richieste, ovvero \emph{SPAM\_Sensor}. Una volta aggiunto allo
schema basterà configurarlo aprendo le proprietà del blocco ed utilizzare i
valori come descritto nella sezione \ref{sec:SPAMSensS}. Una volta ripetuti
tutti questi passaggi basterà collegare tutte le porte restanti del blocco
di \foreignword{dispatch} con un blocco contentente $[0;0;0]$(Immagine
\ref{img:Tut03}).

\image  {Images/Tutorial/Step03}
        {Inserimento dei blocchi \emph{SPAM\_Sensor}}
        {.8}
        {img:Tut03}

Siccome \SciCos{} simula dei sistemi discreti sarà necessario collegare i
blocchi che effettuano le comunicazioni ad un timer in quanto, altrimenti,
non avverrebbe una comunicazione continua. Non bisogna preoccuparsi di una
possibile desincronizzazione tra il clock ed il blocco di comunicazione,
visto che le chiamate usate nei blocchi sono bloccanti, quindi finchè non
viene ricevuta una risposta il sistema non chiama i blocchi successivi, tra
i quali figurano quelli di clock (Immagine \ref{img:Tut04}

\image  {Images/Tutorial/Step04}
        {Collegamento di un clock al blocco \emph{SENS\_Disp}}
        {.8}
        {img:Tut04}

Le porte in uscita, come già detto, conterranno i dati ricevuti dallo
\SPAM{} in base alle richieste settate sulle porte in entrata.

Per quanto riguarda il controllo dei motori, invece, c'è una leggera
differenza, ovvero come gestire i parametri che vanno inviati ai motori. Il
procedimento è del tutto simile a quello per i sensori, con la differenza
che, al posto di un blocco \emph{SENS\_Disp} serve un blocco
\emph{MOTOR\_Disp}, il quale va configurato tramite l'uso di blocchi
\emph{SPAM\_Motor} (Immagine \ref{img:Tut05}.

\image  {Images/Tutorial/Step05}
        {L'utilizzo dei blocchi per il controllo dei motori}
        {.8}
        {img:Tut05}

Come si potrà notare dall'immagine i blocchi di codifica dei motori hanno
una porta di input alla quale collegare il parametro da inviare ai motori.

\section{Far partire la simulazione}
Una volta pronti a far partire la simulazione bisogna far partire il server
collegandolo allo \SPAM{}, sul quale bisognerà aver fatto partire
l'applicazione sviluppata appositamente.

Per fare questo basta andare nella cartella del server e farlo partire con
il seguente comando:
\begin{verbatim}
./BROFist --mac=01:23:45:67:89:AB
\end{verbatim}
Dove il valore dopo l'opzione \cypher{--mac} è il \emph{MAC Address}
dello \SPAM{} al quale connettersi. Una volta connesso allo
\SPAM{}\footnote{Cosa che viene indicata sul display dello \SPAM{} con un
\cypher{[BT]}} sulla linea di comando comparirà la scritta
\begin{verbatim}
Ready for client connect().
\end{verbatim}
Adesso si può far partire la simulazione sa \SciCos{}. Una volta terminata
il server si chiuderà da solo e bisognerà riavviare lo \SPAM{} premendo il
tasto verso sinistra, riportandolo alla schermata di \nxtOSEK{}.

\section{Le opzioni del server per BROFist}
Il server ha varie opzioni, le quali possono venir lette utilizzando
l'opzione \cypher{-h} quando si esegue il server.

Le opzioni disponibili sono le seguenti:
\begin{itemize}
    \item \cypher{-m \{Mac Address\}} oppure {--mac=\{Mac Address\}}:
        Connettersi al \emph{MAC Address} specificato, il quale dovrà venir
        inserito nella forma \emph{XX:XX:XX:XX:XX:XX}.
    \item \cypher{-l} oppure \cypher{--list-devices}: Fare uno scan
        \techname{BlueTooth} e stampare a video tutti i device in range,
        indicandone nome e \emph{MAC Address}
    \item \cypher{-s \{BlueTooth Name\}} oppure
        \cypher{--select-device=\{BlueTooth Name\}}: Connettersi al device
        \techname{BlueTooth} con il nome fornito. Se sono presenti più
        device con lo stesso nome il server si connetterà al primo
        disponibile.
    \item \cypher{-h} oppure \cypher{--help}: Stampare a video l'help del
        server, il quale riassume le opzioni disponibili.
\end{itemize}


\cleardoublepage

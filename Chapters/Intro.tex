\chapter{Introduzione}
\label{chap:Intro}

\section{I sistemi Embedded}
\graffito{I sistemi Embedded nella vita di tutti i giorni}
Negli ultimi tempi le vere rivoluzioni tecnologiche si sono sempre
dimostrate quelle riguardanti l'integrazione di device nella vita di tutti
i giorni senza che le persone che ne usufruiscono si accorgano della loro
presenza. Basta pensare alla enorme quantit\'a di device con la quale ci
siamo abituati a convivere, quali navigatori satellitari che stanno nel
palmo della mano oppure integrati all'interno degli smartphone, ormai
potenti quanto PC, oppure ancora alla presenza di sistemi di controllo
elettronici per la stabilit\'a all'interno delle macchine.

Tutti questi sistemi fanno parte di una categoria di sistemi
microprogrammati creati specificatamente per eseguire una oppure un numero
limitato di funzioni: i sistemi Embedded. Questi spaziano dalla robotica
all'Entertainment portatile, dalla medicina all'applicazione in campo
militare. \'E per questo che si fa sempre pi\'u necessario avere competenze
in questo campo, in quanto la richiesta \'e destinata ad aumentare,
come aumenter\'a la necessit\'a di avere strumentazioni tecnologiche sempre
pi\'u integrate nella vita di tutti i giorni, senza che l'utente finale si
accorga della loro presenza.

Programmare applicazioni per questi sistemi pu\'o essere
estremamente complicato, soprattutto per coloro che sono meno esperti nella
programmazione o che sono abituati a produrre software per piattaforme di
alto livello.

\graffito{Le difficolt\'a derivanti dalla programmazione di software per sistemi
Embedded} Il problema derivante dalla programmazzione di software per
sistemi Embedded \'e la necessit\'a di lavorare con risorse limitate.
Questo non permette una grande astrazione del codice, richiedendo al
programmatore delle competenze specifiche in linguaggi di basso livello.
Inoltre il debugging di questo tipo di piattaforme pu\'o essere complicato
o, in alcuni casi, impossibile visto che potrebbero non essere disponibili
periferiche di comunicazione con il programmatore oppure metodi per
recuperare file di log direttamente dalla memoria del sistema.

Siccome molto spesso pu\'o essere un'attivit\'a che richiede molto tempo
conseguire le abilit\'a necessarie per programmare in maniera corretta su
piattaforme di questo tipo \'e utile avere una piattaforma di sviluppo
sulla quale formare le proprie competenze in questo campo per poi passare
allo sviluppo a basso livello.

\graffito{L'idea alla base}
L'idea, dunque, \'e quella di creare un set di strumenti atto allo
spostamento dello sviluppo di software per piattaforme Embedded su un
applicazione di modellazione e simulazione di sistemi dinamici ed avere un
feedback istantaneo. Questo differisce dal metodo tradizionale di modellazione
in quanto permette l'utilizzo di dati influenzati da fattori esterni, quali
la carica della batteria dell'unit\'a, l'attrito intrinseco dei motori, il
quale varia da motore a motore, la presenza di ostacoli e cos\'i via.

Questo permetterebbe, quindi, uno sviluppo di software basato su \emph{real
case scenarios} invece che su simulazioni in ambienti protetti dove l'unico
modo per rendere pi\'u realistica la simulazione sarebbe introdurre degli
elementi esterni ``fittizi'', oppure scrivere il codice relativo al modello
creato, testarlo sulla piattaforma Embedded registrando tutti i dati per
una seconda analisi e quindi modificare il modello di conseguenza.

\graffito{Scegliere gli strumenti}Per questo scopo la scelta di strumenti
di sviluppo corretti \'e un punto cruciale. Sono due le scelte che vanno
fatte, ovvero la piattaforma \emph{hardware}, quella \emph{software} che
poi dovr\'a venir installata sulla prima ed, infine, un \emph{pacchetto di
modellazione e simulazione di sistemi dinamici}.

Inoltre, una volta effettuate tutte queste scelte, andr\'a sviluppato un
sistema per l'interfacciamento della piattaforma Embedded con il software
di simulazione.

\section{La piattaforma Hardware}
La prima scelta da effettuare, dunque, \'e quale piattaforma hardware
utilizzare per apprendere delle competenze di base nella programmazione
Embedded.

L'idea che sta alla base dei sistemi Embedded \'e, come detto sopra, che le
piattaforme hardware vengono progettate per un compito specifico, quindi
progettare specificatamente una piattaforma per questo scopo sarebbe una
spesa inutilmente grande.

\subsection{Lego MINDSTORMS NXT}
La scelta della piattaforma hardware non \'e stata una procedura lunga in
quanto sul commercio esiste una piattaforma di tipo Embedded con un costo
relativamente contenuto e di ampia distribuzione: i \nxt.

Questa \'e una linea di prodotti della Lego
ideata per la costruzione di robot e altri sistemi automatici o
interattivi. \graffito{ATMEL AT91SAM7S256} Il nucleo di controllo per i
\nxt{} \'e un micro computer basato sull'\emph{AT91SAM7S256}, un
microcontrollore multi-purpose della \emph{ATMEL}. Questa piattaforma
dispone, oltre che ad una connessione USB, di controller Bluetooth, I2C ed
RS485.

La piattaforma \emph{AT91SAM7S256} \'e basata su un processore
\emph{ARM7}, utilizzato gi\'a in vari prodotti di grande distribuzione,
quali:
\graffito{L'uso del processore ARM7 sul mercato}
\begin{itemize}
    \item Il \emph{Sega Dreamcast}
    \item L'\emph{Apple iPod}
    \item Il \emph{Nintendo DS} ed il \emph{Nintendo GBA}
    \item La maggior parte dei telefoni portatili della \emph{Nokia}
    \item Una gran quantit\'a di modelli di automobili.
\end{itemize}
Dunque questa piattaforma, oltre ad essere facilmente reperibile, presenta
anche un esempio di hardware che potrebbe venir realmente usato per lo
sviluppo di device Embedded. Inoltre i \nxt{} vengono utilizzati spesso
nelle scuole ed in luoghi come i musei di scienze per scopi didattici,
anche grazie all'interfaccia di programmazione visiva. Questa interfaccia,
per\'o risulta essere estremamente limitata appena bisogna sviluppare
qualcosa di complesso, come un client per la comunicazione
\emph{BlueTooth}.

\subsubsection{Le altre interfaccie I/O disponibili per gli NXT}
Un altro punto forte dei \nxt{} \'e la disponibilit\'a di una vasta gamma
di sensori ``di serie'' inclusi all'interno delle scatole base, con i quali
poter interagire con il mondo esterno, oltre a quelle fornite dalla
piattaforma \emph{AT91SAM7S256} 

\section{La piattaforma Software}
Una volta deciso su quale piattaforma hardware operare la scelta successiva
era quella di quale sistema software montare su di essa per avere il
controllo corretto dei sensori e dei servo motori. Le scelte vagliate sono
state fondamentalmente tre:
\begin{description}
    \item[nxOS]\cite{bib:nxos}un sistema operativo per \nxt{} scritto da David Anderson,
    il quale, per\'o non sta pi\'u venendo aggiornato da maggio del 2009
    \item[NBC \& NXC]\cite{bib:nbcnxc}due ``dialetti'' derivati, il primo,
    dall'\emph{Assembly} e, il secondo, dal \emph{C}.
    \item[nxtOSEK]un sistema operativo basato su specifiche \emph{OSEK}
    sviluppato da Takashi Chikamasa.
\end{description}
Dopo una lunga ricerca e svariati test effettuati su queste piattaforme
Software la scelta \'e ricaduta sul sistema \emph{nxtOSEK}.

\subsection{nxtOSEK}
\nxtOSEK{}\cite{bib:nxtosek} \'e una piattaforma open source per i \nxt{} formata da tre
parti:
\begin{description}
    \item[Device Drivers]in C/Assembly di \emph{leJOS NXJ}, un firmware di
        rimpiazzo per i \nxt{} basato su \emph{Java}
    \item[TOPPERS/ATK]conosciuto come \emph{TOPPERS/OSEK}, visto che
        segue le specifiche \emph{OSEK}\cite{bib:osek}, uno standard per sistemi
        operativi, stack di comunicazione e protocolli di rete per sistemi
        embedded. Questo fornisce un ambiente real-time multitaking che
        segue queste specifiche.
    \item[TOPPERS/JSP]fornisce un ambiente real-time multitasking basato
        sulle specifiche giapponesi \emph{$\mu$ITRON}\cite{bib:muitron}.
\end{description}
Grazie all'unione di queste features e all'utilizzo di una toolchain GCC
per ARM \emph{nxtOSEK} fornisce un ambiente di sviluppo Real Time in Ansi C
/ C++.

Tutto questo assieme si rivela essere la miglior scelta, in quanto, oltre
a non necessitare di un periodo di training per imprare un linguaggio
specifico per sviluppare il client per la comunicazione con l'ambiente di
sviluppo grafico, permetter\'a, in futuro, di utilizzare sistemi di
generazione automatica di codice coi quali trasformare dei modelli visivi
direttamente in applicazioni per \emph{nxtOSEK}.

\section{Ambiente di Sviluppo Grafico}
Come detto sopra la programmazione di basso livello su piattaforme
Embedded \'e complicata e, soprattutto, richiede un gran numero di trial
and error con un feedback minimo da parte della piattaforma. Per questo si
rende necessario un sistema pi\'u facile e pi\'u intuitivo per la
realizzazione di applicazioni su sistemi Embedded, oltre che ad un sistema
per il logging di ci\'o che accade sull'unit\'a NXT.

\subsection{Strumento di Modellazione e Simulazione}
In un gran numero di ambienti di ricerca per la simulazione e la
progettazione di sistemi Embedded viene utilizzato
\emph{Matlab}, un ambiente per la computazione numerica, assieme a \emph
{Simulink}, un pacchetto per la modellazione e la simulazione di sistemi
Dinamici ed Embedded.

Questo permette a persone esperte nella
modellizzazione di sistemi elettronici di creare dei modelli teorici di
software per piattaforme Embedded e poi testarli senza dover scrivere il
codice, eliminando la necessit\'a di avere delle competenze da
programmatori.

\graffito{L'uso di SciCosLab}
Esiste comunque un'alternativa OpenSource a Matlab, ovvero \emph{SciCosLab},
il quale \'e a sua volta un pacchetto di modellazione e simulazione di
sistemi dinamici. Siccome tutto il progetto faceva uso
di strumenti OpenSource \'e stato deciso di mantenere questa linea fino in
fondo.

\section{Il sistema di Logging}
In \nxtOSEK{} (O sugli altri sistemi testati, per quato ho potuto
constatare) non \'e possibile scrivere files sulla memoria dell'unit\'a per
poi recuperarli in futuro ed utilizzare i dati di logging per debug oppure
per fine tuning del sistema. \'E Per questo che si rende necessario lo
sviluppo di un protocollo di comunicazione via BlueTooth col quale
richiedere i dati letti dai sensori per utilizzarli a fine di \emph{debug},
\emph{Fine tuning} oppure \emph{controllo diretto utilizzando i dati
ricevuti}.

\graffito{BROFist}Oltre a questo era necessario incorporare in questo
protocollo anche una codifica per il controllo dei servo motori, cos\'i da
poter controllare completamente (per quanto riguarda motori e sensori,
ovviamente) l'unit\'a NXT da remoto.

A questo protocollo ho dato il nome di \BROFist{}: BROFist - Robot
Operational Fist, che sta ad indicare il rapporto di fratellanza che si
crea tra Robot e Computer grazie al suo utilizzo.

\subsection{Il problema del controllo remoto via BlueTooth}
Bisogna ammettere, per\'o, che questo sistema ha una debolezza. A causa di
una gestione non molto ``furba'' del \emph{DMA} sull'unit\'a NXT e di un
device per il controllo del BlueTooth non efficente la latenza delle
comunicazioni rasenta i 45ms.

Una latenza di questo tipo non sarebbe un problema se si volesse utilizzare
il sistema solo per il plotting o per il logging dei dati letti dai
sensori, per\'o pu\'o essere problematica in caso di implementazione di
\emph{PID Controller} o altri sistemi di controllo digitale che richiedono
una responsivit\'a maggiore.

\section{Il BROFist Client per lo SPAM}\label{sec:BROClient}
Una delle parti del progetto, dunque, \'e stata quella di scrivere il
client per l'interpretazione dei \emph{BROFists}, oltre che di controllo
della velocit\'a dei motori.

\graffito{Struttura del BROFistClient}
Il client \'e formato da quattro task, ognuno con il suo compito specifico:
\begin{description}
    \item[Comunicazioni Bluetooth e Decodifica dei BROFists]utilizzando
        le definizioni per il protocollo condivise tra tutte le parti del
        progetto. Questo \'e, fondamentalmente, il nucleo dell'intero
        Client: recupera i dati richiesti dai vari sensori e setta le
        potenze o le velocit\'a da raggiungere per i singoli Servo Motori.
    \item[Calcolo delle velocit\'a istantanee dei motori]il quale legge
        periodicamente i dati dei sensori Tachimetrici e li registra
        all'interno delle strutture contenenti i dati dei singoli Servo
        Motori.
    \item[Controllo della velocit\'a dei motori]tramite l'utilizzo di un
        \PID{}, il quale usa i dati relativi alle
        potenze ed agli errori di velocit\'a per calcolare quanta potenza
        utilizzare per far andare i motori alla velocit\'a richiesta.
    \item[Rappresentazione di dati sullo schermo]utilizzando un comando
        delle API di \nxtOSEK{}. Naturalmente questi dati possono venir
        configurati modificando il codice sorgente del client.
\end{description}
Ogni task viene fatto girare in contemporanea, grazie alla struttura di
\nxtOSEK{}, il quale permette l'esecuzione multitask realtime. 
\graffito{La necessit\'a di un sistema Realtime}
Questo significa inoltre che sappiamo a priori che i vari task avranno un tempo di
esecuzione massimo ben specifico.
Siccome i tempi di aggiornamento delle velocit\'a e gli intervalli di
aggiornamento delle potenze da applicare ai motori devono essere sempre gli
stessi \'e necessario che il sistema segua le specifiche che gli vengono
fornite, ed \'e per questo che il sistema \nxtOSEK{} \'e stato scelto come
base per il Client.

\cleardoublepage

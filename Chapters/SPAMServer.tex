\chapter{Il Server di BROFist per PC}
\label{chap:PCServ}

In questo capitolo parleremo dello sviluppo del Sever per PC
del progetto \BROFist{} e delle librerie utilizzate.

\section{La struttura di BROFist}
Il progetto \BROFist{} è formato da tre parti ben distinte tra di loro:

\begin{description}
    \item[Un Client su NXT]il quale deve ricevere degli ``ordini'' ed
        eseguirli per poi inviare al Server i risultati di questi ordini.
    \item[Un Server su PC]il quale riceve degli ordini per lo \SPAM{} da
        parte dell'interfaccia grafica ed invia questi ordini al client in
        esecuzione sull'unità.
        
        Una volta ricevuta la risposta dal Client questa viene inoltrata
        all'interfaccia.
    \item[Dei blocchi d'interfaccia per SciCos]i quali si collegano al
        server, grazie al quale possono comunicare con lo \SPAM{} ed
        utilizzare i dati ricevuti nelle simulazioni.
\end{description}

Questa schematizzazione, però, è frutto di una lunga progettazione, la
quale è stata necessaria per creare il sistema più flessibile possibile,
oltre ad oltrepassare delle limitazioni del pacchetto grafico di
simulazione e modellazione \techname{SciCosLab}.

\subsection{La struttura di BROFist nel tempo}
\graffito{La prima versione}
L'idea iniziale era di creare semplicemente un logger che inviava in
realtime vari dati calcolati dal software in esecuzione sullo \SPAM{}, i
quali venivano ricevuti da un server scritto in \techname{Python}.

Il problema di questa struttura, però, era la mancanza di flessibilità, in
quanto, ogni volta che venivano cambiati i dati inviati, bisognava cambiare
il codice del server. Questo perchè il metodo utilizzato per
l'interpretazione dei dati ricevuti via \techname{BlueTooth} prevedeva
l'utilizzo della funzione \codeconst{struct.unpack}, la quale richiede una
stringa che indica la struttura dei dati da interpretare.

\graffito{La decisione di sviluppare un protocollo}
Per risolvere questo problema si è quindi deciso di sviluppare un sistema
che facesse uso di un protocollo per la richiesta di dati dei sensori e per
il controllo da remoto dei motori.

Questo, però, richiedeva lo sviluppo di un sistema che, ricevute le
richieste, si comportasse in accorodo con esse. \'E per questo che il
sistema è passato da un'applicazione che inviava dati ad un PC ad un
modello \techname{Client-Server}.

Per creare il Server, dunque, si è deciso di utilizzare \techname{C} invece
che \techname{Python}, in quanto c'era bisogno di una gestione delle
strutture a basso livello e di prestazioni migliori\cite{bib:progbench}.

La scelta di passare a \techname{C}, però, ha richiesto del tempo per
arrivare alle competenze necessarie\footnote{La quale, però, che è stata
facilitata grazie alla documentazione presente online
\cite{bib:blueZdocs}}.

\graffito{L'idea di creare dei blocchi d'interfaccia per SciCosLab}
Vista la possibilità di comunicare via \techname{BlueTooth} utilizzando
\techname{C} invece di \techname{Python} si è reso automaticamente
disponibile lo sviluppo di blocchi di codice per \techname{SciCosLab}.
Questo ha reso necessario lo strutturare il progetto in modo che
funzionasse tra blocchi di \techname{SciCosLab} ed il Client su \SPAM{}.

Dunque l'idea iniziale era quella di sviluppare il progetto in direzione di
un modello \emph{Client-Server} tra \techname{SciCosLab} ed \techname{NXT}.
Purtoppo, però, il pacchetto di modellazione \techname{SciCos} integrato in
\techname{SciCosLab} manca della flessibilità necessaria\footnote{Si faccia
riferimento al paragrafo \ref{sec:whyitsucks} per capire meglio quali sono
i problemi derivanti dallo sviluppo di blocchi in \techname{C} per
\techname{SciCos}} per sviluppare questo progetto.

\graffito{Il modello SciCosLab - Server - SPAM Client}
\'E grazie a questa serie di tentativi e di modelli che si è arrivati
all'idea finale per il progetto \BROFist{}, la quale prevede l'utilizzo di
blocchi d'interfaccia per \techname{SciCosLab}, i quali comunicazno con un
server che gira sullo stesso computer\footnote{Anche se, grazie alla
struttura data al progetto, non è escluso che si possa utilizzare un server
in remoto}, connesso ad uno \SPAM{} via \techname{BlueTooth}.

Grazie al questa struttura, dunque, abbiamo un sistema flessibile che
permette la sua estensione in base alle necessità dell'utente.

\section{Le ragioni dello sviluppo di un Server}
Abbiamo già detto che una delle motivazioni per la scelta di un modello
basato su server era la poca flessibilità data dallo sviluppo di blocchi
per \techname{SciCos}. Ovviamente questo non è stato l'unico fattore che ha
portato alla scelta fatta.

Il punto forte di questo sistema è l'estensibilità data dall'uso dei
\techname{Socket UNIX} e dal fatto che il codice utilizzato per comunicare
con l'\techname{NXT} è separato dall'applicazione che la necessita. La
schematizzazione di questa struttura e delle possibili estensioni future è
rappresentata nell'immagine \ref{img:BROFistStruct}

\image  {Images/TheBROFistStructure}
        {La struttura del progetto \BROFist{}, con i possibili sviluppi
        futuri, segnati come delle linee viola tratteggiate}
        {.8}
        {img:BROFistStruct}

Questo permette, conoscendo le strutture utilizzate per la comunicazione,
di sviluppare applicazioni di qualunque tipo che sfruttano il server per
comunicare con un'unità \SPAM{}, controllarne i motori oppure ricevere dati
dei vari sensori.

\graffito{Esempio di applicazione alternativa per BROFist: Web Interface}
Mettiamo, ad esempio, di voler creare un esperimento sociale attreverso la
rete dove la gente può prenotare l'utilizzo di un'unità \SPAM{} alla quale
è stata aggiunta una Webcam. Questa unità è stata posta all'interno di uno
spazio chiuso con vari oggetti coi quali interagire, ostacoli ed altro.

Per sviluppare questo progetto sarebbe necessaria un'interfaccia Web la
quale si collega via remoto ad un server \BROFist{} il quale manderebbe i
pacchetti di ordini allo \SPAM{} e spedirebbe indietro i dati letti dai
sensori, oltre che alle immagini della Webcam\footnote{Anche se i dati
della Webcam verrebbero gestiti dall'interfaccia Web.}.

\graffito{Gli altri sviluppi resi possibili dall'uso di un server}
Oltre alla possibilità di creare applicazioni ``ad hoc'' che usano il
server \BROFist{} questo tipo modello ci permette anche di estendere i
metodi di connessione del server allo \SPAM{}.

Per ora il sistema comunica solo via \techname{BlueTooth} però uno dei
possibili sviluppi futuri\footnote{Per quanto sia complesso utilizzare
l'\techname{USB}, visto che il supporto LibUSB è stato rimosso da
\nxtOSEK{} dalla release del 2009} potrebbe essere l'abilitazione delle
comunicazioni via \techname{USB} con l'unità \techname{NXT}.

Inoltre altre aggiunte potrebbero venir applicate a Client e Server per
permettere la comunicazione dei contenuti di variabili ben definite, oppure
per il controllo delle periferiche collegate alla porta \techname{RS485}.

\section{Il funzionamento del server}
Il server è diviso in quattro moduli separati, ognuna con delle funzioni
ben specifiche. Le quattro parti separate sono:

\begin{description}
    \item[Controllo BlueTooth]che contiene tutte le funzioni per la
        connessione e la comunicazione attraverso \techname{BlueTooth}
    \item[Controllo Socket UNIX]che contiene tutte le funzioni per la
        gestione del server, la comunicazione via \techname{Socket UNIX} e
        la sua interazione con le comunicazioni \techname{BlueTooth}
    \item[Parsing delle Opzioni]che si occupa di fare il \emph{parsing}
        delle opzioni e quindi comportarsi di conseguenza.
    \item[Il Server BROFist]il quale utilizza i dati settati dal Parser per
        iniziare le comunicazioni, fare il forwarding dei dati e quindi
        bloccarsi quando riceve l'ordine di stop.
\end{description}

Vedremo ora le varie funzionalità di ogni modulo, parlando anche delle
strutture utilizzate.

\subsection[bro\_bt]{bro\_bt: Gestione BlueTooth}
In questo modulo sono contenute tutte le funzioni per la gestione del
\techname{BlueTooth} per il server di \BROFist{}. Sono inoltre contenute la
dichiarazione della struttura per l'identificazione dei device
\techname{BlueTooth} e le defines per la discovery e per la connessione dei
devices.

La scrittura di questo modulo, come quello del controllo delle
comunicazioni via \techname{Socket UNIX}, ha richiesto del tempo per
comprendere il funzionamento delle \emph{API} di \techname{BlueZ}, il
\emph{Protocol Stack} ufficiale per Linux.

Le funzioni di questo modulo sono:
\begin{itemize}
    \item \codeconst{size_t bro_bt_scan_devices (bro_bt_device_t *)}, la
        quale serve per la \emph{Discovery} dei device \techname{BlueTooth}
        in range, inserendoli all'interno di un array di strutture di tipo
        \codeconst{bro_bt_device_t}.

        Questo serve sia per stampare a video la lista di device, sia per
        collegarsi ad un device sapendo il suo nome.
    \item \codeconst{int bro_bt_connect_device (int *, bdaddr_t)}, funzione
        che serve a connettere un device, conoscendone il \emph{MAC
        Address}. L'indirizzo dev'essere passato sotto forma di codifica a
        $6$ bytes, come richiesto dalla libreria
        \techname{BlueZ}\footnote{Operazione che potrebbe risultare
        complicata. Fortunatamente \techname{BlueZ} fornisce anche la
        funzione \codeconst{int str2ba(const char *, bdaddr_t * )}, la
        quale converte un \emph{MAC Address} da formato stringa a formato a
        $6$ bytes.}.
        
        La connessione viene quindi legata ad un \emph{Socket
        Descriptor}, che verrà utilizzato dal programma per comunicare con
        lo \SPAM{}
    \item \codeconst{int bro_bt_client_fist (bro_fist_t *, bro_fist_t
        *,int)}, funzione che invia un set di \emph{BROFists} ed aspetta la
        risposta. Una volta ricevuti i dati dallo \SPAM{} la funzione
        riempie un set di \emph{BROFists} di egual misura di quelli
        inviati contenenti i dati richiesti.
    \item \codeconst{int bro_bt_close_connection{int}}, la quale chiude la
        comunicazione ed il socket descriptor ad essa collegato.
\end{itemize}

In questo modulo, inoltre, è presente la dichiarazione della struttura per
rappresentare un device \techname{BlueTooth} all'interno del progetto
\BROFist{}, la quale è rappresentata nel listato \ref{lst:BROToothS}

\begin{figure*}[htbp]
    \begin{lstlisting}[caption=Sample Task-Alarm declaration,
                       label=lst:BROToothS]
typedef struct {
    /** Human Readable Name (As set on the Device). */
    char       name[248];      
    /** Mac address stored in a bdaddr_t variable */
    bdaddr_t   mac;             
} bro_bt_device_t;

    \end{lstlisting}
\end{figure*}

\subsection[bro\_comm]{bro\_comm: Gestione Comunicazioni via Socket UNIX}
Forse il modulo più importante del progetto, \emph{bro\_comm}, contiene le
funzioni per il controllo dell'\emph{IPC Socket} per la comunicazione con
le applicazioni sul PC, oltre che all'interazione tra \emph{IPC Socket} e
\emph{BlueTooth Socket}.

Le funzioni presenti in questo modulo sono solo tre:
\begin{itemize}
    \item \codeconst{int bro_start_server (int *, int *)} la quale
        inizializza il server, aspetta una connessione da un client che
        utilizza gli \emph{IPC Sockets} e quindi fa il \emph{Bind} di
        queste due cose a due \emph{socket descriptors}
    \item \codeconst{int bro_server_fist (const bro_fist_t *, bro_fist_t *,
        int, int} è la funzione principale dell'intero server. Questa
        attua la comunicazione tra l'applicazione che richiede l'utilizzo
        di \BROFist{} e lo \SPAM{}.
    \item \codeconst{bro_stop_server (int, int)} è la funzione che chiude
        le comunicazioni via \emph{IPC Socket} ed i socket ad esse
        collegate.
\end{itemize}

\subsection[bro\_fist]{bro\_fist: Server e dichiarazioni delle strutture
utilizzate in tutto il progetto BROFist}
Le dichiarazioni pubbliche di questo module vengono utilizzate in tutto il
progetto\footnote{Inclusi blocchi per \techname{SciCosLab} e tasks per
\nxt{}}, in quanto sono quelle di come sono strutturati i \emph{BROFists},
le codifiche dei comandi, la posizione del \emph{file descriptor} per la
connessione al server da parte di altre applicazioni ed il numero di
\emph{BROFists} inviati per comunicazione.

Questo modulo non contiene funzioni pubbliche, in quanto il file
\techname{.c} ad esso collegato fa solo da \emph{main} per il server.

Siccome abbiamo già parlato della struttura dei \emph{BROFists} nel paragrafo
\ref{sec:BROStruct} quì metteremo il codice utilizzato per 

\section{Dipendenze funzionali del Server}


\cleardoublepage

\chapter{Il Server di BROFist per PC}
\label{chap:PCServ}

In questo capitolo parleremo dello sviluppo del Sever per PC
del progetto \BROFist{} e delle librerie utilizzate.

\section{La struttura di BROFist}
Il progetto \BROFist{} è formato da tre parti ben distinte tra di loro:

\begin{description}
    \item[Un Client su NXT]il quale deve ricevere degli ``ordini'' ed
        eseguirli per poi inviare al Server i risultati di questi ordini.
    \item[Un Server su PC]il quale riceve degli ordini per lo \SPAM{} da
        parte dell'interfaccia grafica ed invia questi ordini al client in
        esecuzione sull'unità.
        
        Una volta ricevuta la risposta dal Client questa viene inoltrata
        all'interfaccia.
    \item[Dei blocchi d'interfaccia per SciCos]i quali si collegano al
        server, grazie al quale possono comunicare con lo \SPAM{} ed
        utilizzare i dati ricevuti nelle simulazioni.
\end{description}

Questa schematizzazione, però, è frutto di una lunga progettazione, la
quale è stata necessaria per creare il sistema più flessibile possibile,
oltre ad oltrepassare delle limitazioni del pacchetto grafico di
simulazione e modellazione \techname{SciCosLab}.

\subsection{La struttura di BROFist nel tempo}
\graffito{La prima versione}
L'idea iniziale era di creare semplicemente un logger che inviava in
realtime vari dati calcolati dal software in esecuzione sullo \SPAM{}, i
quali venivano ricevuti da un server scritto in \techname{Python}.

Il problema di questa struttura, però, era la mancanza di flessibilità, in
quanto, ogni volta che venivano cambiati i dati inviati, bisognava cambiare
il codice del server. Questo perchè il metodo utilizzato per
l'interpretazione dei dati ricevuti via \techname{BlueTooth} prevedeva
l'utilizzo della funzione \codeconst{struct.unpack}, la quale richiede una
stringa che indica la struttura dei dati da interpretare.

\graffito{La decisione di sviluppare un protocollo}
Per risolvere questo problema si è quindi deciso di sviluppare un sistema
che facesse uso di un protocollo per la richiesta di dati dei sensori e per
il controllo da remoto dei motori.

\cleardoublepage

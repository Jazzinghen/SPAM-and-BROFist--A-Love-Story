\chapter{Compilare BROFist}
In questo capitolo vedremo velocemente come fare per compilare ogni parte
del progetto e delle librerie necessarie per questa operazione.

\section{BROFist Server}
La compilazione del server, una volta installate le dipendenze è una cosa
facile quanto fare
\begin{verbatim}
cd src
make all
\end{verbatim}
Verrà creato un eseguibile chiamato \cypher{BROFist} nella cartella ed il
server sarà pronto per l'utilizzo.

\section{Blocchi per SciCos}
Per i blocchi di \SciCosLab{} la storia potrebbe essere semplice come per
il server, visto che ho incluso un file di script che contiene tutte le
operazioni che vanno fatte per preparare l'ambiente, come visto nella
sezione \ref{sec:SciComp}, però c'è un problema dato dalla mancanza di
flessibilità di \SciCosLab{}.

Come si può notare leggendo il file di script \cypher{BRO\_Loader.sce} la
prima riga contiene il seguente ``comando'':
\begin{verbatim}
base = "/home/jazzinghen/Documents/Tesi/BROFist/scicos/";
\end{verbatim}
Il quale indica dove si trovano i codici sorgente e le definizioni dei
singoli blocchi. Purtroppo non ho trovato maniera più ``pulita'' di
scriverlo. Il problema derviva dal fatto che \SciCosLab{} setta come
cartella di lavoro quella dalla quale è stato eseguito, perciò sarebbe
necessario ogni volta settare il \foreignword{Working Path} per compilare
tutto, oppure eseguire \SciCosLab{} ogni volta dalla cartella corretta.

Per facilità, dunque, basta modificare il file \cypher{BRO\_Loader.sce} e
modificare il valore di \cypher{base} con la cartella dove sono presenti i
files per \SciCosLab{} e quindi eseguire \cypher{exec \{Path del progetto
BROFist\}\\scicos\\BRO\_Loader.sce}.

\section{Immagine per NXT}
Anche la compilazione di \nxtOSEK{} potrebbe dare dei problemi. Una volta
che è stato settato l'ambiente di compilazione di \nxtOSEK{} bisognerà
indicare all'intero del \cypher{Makefile} la posizione della configurazione
per la compilazione di \nxtOSEK{}. Per fare ciò basterà modificare la riga
\begin{verbatim}
include ../../spamOSEK/ecrobot/ecrobot.mak
\end{verbatim}

Una volta settato questo bisognerà fare
\begin{verbatim}
cd spam
make all
sh ./rxeflash.sh
\end{verbatim}
Per compilare l'immagine e caricarla sull'unità \techname{NXT}.

\section{Dipendenze}
Il progetto necessità di tre cose, di cui le prime due sono ovvie:
\begin{itemize}
    \item \nxtOSEK{}
    \item \SciCosLab{}
    \item \techname{LibBluetooth}
\end{itemize}
L'ultimo è sempre presente in un sistema \techname{*NIX} in quanto è il suo
\foreignword{Protocol Stack} ufficiale per \techname{BlueTooth}. Per la
compilazione, però, è necesario disporre della versione \emph{Dev} del
pacchetto.

\cleardoublepage

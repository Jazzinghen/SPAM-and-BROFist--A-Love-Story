\chapter{I blocchi d'interfaccia per SciCos}
\label{chap:SCICOSB}

In questo capitolo parleremo dei Blocchi d'interfaccia per \SciCos{}, di
come funzionano e di come interagiscono tra di loro.

Verrà inoltre posta dell'attenzione ai vari problemi che sono insorti
durante il loro sviluppo, oltre che ai problemi dati dall'interfaccia
utilizzata.

\section{Struttura ed interazione dei blocchi}
Come per tutte le parti del progetto anche la progettazione e la
realizzazione dei blocchi ha richiesto una lunga serie di prototipi e di
modelli d'interazione tra i blocchi stessi.

\graffito{Funzionalità di ogni Blocco}
Alla fine si è deciso di dividere il progetto nello schema quì indicato:
\begin{itemize}
    \item Un blocco che controlla la comunicazione con il server attraverso
        gli \emph{IPC Sockets} e passa ai blocchi che attuano la
        comunicazione il \emph{Socket Descriptor} del server
    \item Un blocco generico che codifica, in base all'header
        \cypher{bro\_fist.h} la richiesta per i valori letti da un sensore
        in base ai dati inseriti dall'utente.
    \item Un blocco generico che codifica, sempre in base allo stesso file
        sopra citato, i comandi per attuare i Servo Motori.
    \item Un blocco che controlla l'invio soltanto dei \emph{BROFists} che
        richiedono dati dai sensori e della distribuzione dei dati ricevuti
        in base a queste richieste.
    \item Un blocco che controlla l'invio soltanto dei \emph{BROFists} che
        contengono i dati per l'attuazione dei Servo Motori
\end{itemize}

\graffito{Il perchè della scelta di questa struttura}
Questa struttura permette una personalizzazione dei dati richiesti senza
richiedere un grande sforzo all'utente, il quale può porre attenzione al
modello, invece che al setup per la comunicazione.

\'E anche vero che, a causa della struttura intrinseca di \SciCos{},
l'utilizzo di questi blocchi è più complicato di quanto non potrebbe
essere, ma di questo ne parleremo più approfonditamente nella sezione
\ref{sec:whyitsucks}.

\graffito{Interazioni tra blocchi}
I blocchi per l'interfacciamento devono venir collegati tra loro in una
determinata maniera per funzionare correttamente, in quanto richiedono
specifici dati che solamente gli altri blocchi possono fornir loro.

La porta di output del blocco di controllo delle comunicazioni va collegato
alla prima porta di ogni blocco per l'invio dei \emph{BROFists}, in quanto
attraverso quella porta viene comunicato il \emph{Socket Descriptor} per la
comunicazione col server.

I vari blocchi che inviano \emph{BROFists} hanno una porta (ad esclusione
della prima) per ogni richiesta che possono inviare. A queste porte vanno
collegati i blocchi di codifica richiesti\footnote{Se rimangono delle porte
non utilizzate allora bisogna collegare un blocco con dei dati
\emph{dummy}, ovvero una matrice $1x3$ così formata $[0;0;0]$, ad ognuna
delle porte inutilizzate.}, i quali formeranno i \emph{BROFist} inviati.

Nel caso dei blocchi per la richiesta di dati dei sensori sui blocchi sono
anche presenti delle porte in uscita, una per blocco richiesto, sulle quali
si renderanno disponibili i dati ricevuti dallo \SPAM{}.

Per i codificatori di richieste di motori, oltre ai parametri settati
dall'utente, è necessario collegare alla porta d'input di ogni blocco i
dati da inviare ai motori.

Un esempio di un programma base è mostrato nella figura
\ref{img:SciCosSampProg}.

\image  {Images/SciCosScreen}
        {Un programma d'esempio per \SciCos{} che utilizza i blocchi per la
        comunicazione con uno \SPAM{}}
        {1}
        {img:SciCosSampProg}

Adesso parleremo di ogni singolo blocco e delle funzioni che li fanno
funzionare.

\section[BRO\_Controller]{BRO\_Controller: Blocco di controllo delle
comunicazioni col server di BROFist}
\image  {Images/BROController}
        {Il blocco BRO\_Controller}
        {.2}
        {img:BROCont}
Il blocco \emph{BRO\_Controller} è incaricato di aprire le connessioni con
il server durante l'inizializzazione, di comunicare agli altri blocchi
quale \emph{Socket Descriptor} connettersi e quindi, quando viene chiamata
la finalizzazione della simulazione, di chiudere il socket.

\graffito{Porte del blocco}
Questo blocco ha soltanto una porta in output, la quale conterrà il valore
del \emph{Socket Descriptor}. Questo è stato possibile in quanto i
\emph{Socket UNIX} utilizzano dei descrittori che sono codificati come
\cypher{integer}. Ciò ha permesso di aggirare uno dei problemi presenti
all'interno di \SciCos{}, ovvero l'impossibilità di passare dati differenti
dalla struttura \codeconst{scicos_block} tra il modello grafico e la
funzione \techname{C}.

\graffito{Funzioni collegate al blocco}
La funzione che fa da \emph{wrapper} al blocco è \linebreak \codeconst{void
bro_comm_controller (scicos_block *, int)}, la quale chiama le seguenti
funzioni in base alla flag passatale:
\begin{itemize}
    \item \codeconst{int bro_comm_init(scicos_block *)}, chiamata quando
        viene passata la flag $4$. 

        Questa funzione apre la comunicazione con il server e assegna alla
        porta in output il valore del \emph{Socket UNIX}.
    \item \codeconst{int bro_comm_kill(scicos_block *)}, chiamata quando
        viene passata la flag $5$.

        Questa funzione chiude il socket aperto con il server.
\end{itemize}

L'utilità di questo blocco è data dalla necessità di centralizzare il
controllo del \emph{Socket Descriptor}, evitando così di dover gestire gli
errori di apertura e chiusura dei \emph{Socket} che inevitabilmente
insorgerebbero visto il comportamento di \SciCos{}.

\section[SPAM\_Sensor]{SPAM\_Sensor: Blocco di codifica dei sensori}
\image  {Images/SPAM_Sensor}
        {Il blocco SPAM\_Sensor}
        {.4}
        {img:SPAMSensB}
I blocchi \emph{SPAM\_Sensor} sono incaricati di convertire i dati inseriti
dall'utente, in base ad una codifica specifica per \SciCos{}, in richieste
che seguono il protocollo \BROFist{}.

La funzione collegata al blocco, \codeconst{void bro_motor_enc
(scicos_block *, int)}, utilizza soltanto una funzione, \codeconst{void
bro_motor_enc_run (scicos_block *)}, la quale prende i dati settati come
parametri nella configurazione del blocco e li passa direttamente al blocco
\emph{SENS\_Disp}, il quale convertirà i dati in accordo con le codifiche
presenti nel file \cypher{bro\_fist.h}.

\graffito{Il funzionamento del blocco in \SciCos{}}
Il blocco, inoltre, riesce ad accorgersi della coerenza dei dati settati
nelle sue proprietà.

I valori che possono venir utilizzati per configurare ogni blocco per la
codifica dei sensori sono:
\begin{description}
    \item[1] per selezionare un Contatore Tachometrico
    \item[2] per selezionare la velocità media di un motore
    \item[3] per selezionare un sensore di luuce
    \item[4] per selezionare un sensore di contatto binario
    \item[5] per selezionare un sensore acustico
    \item[6] per selezionare un sensore di distanza ad ultrasuoni
\end{description}

Mentre i valori che possono venir utilizzati per la selezione del motore o
del sensore a cui richiedere il valore sono:
\begin{description}
    \item[1\textasciitilde{}4] per selezionare un sensore posizionato sulla
        porta rispettiva
    \item[5\textasciitilde{}7] per selezionare un motore posizionato su una
        delle porte da \sc{A} a \sc{C}.
\end{description}

Inoltre il blocco aggiorna anche il suo aspetto per rispecchiare la scelta
effettuata dall'utente, facilitando l'utilizzo dei blocchi d'interfaccia.

\section[SENS\_Disp]{SENS\_Disp: Blocco d'invio richieste per i sensori e
ricezione dei dati relativi}
\image  {Images/SENS_Disp}
        {Il blocco SENS\_Disp}
        {.2}
        {img:SENSDisp}

Il blocco \emph{SENS\_Disp} è incaricato di raccogliere fino a sette
richieste da inviare allo \SPAM{} e quindi assegna, quando riceve i dati
dai blocchi, il valore del sensore alla porta relativa.

Questo blocco, come quello per l'invio dei parametri per i motori, richiede
un'attivazione esterna tramite clock a causa della strutturazione modellata
su un sistema discreto, in quanto \SciCosLab{} è un pacchetto per la
modellazione e la simulazione di sistemi discreti.

La funzione \emph{wrapper} di questo blocco, \linebreak \codeconst{void
bro_comm_sens_disp (scicos_block *, int)}, utilizza due funzioni per
l'invio e per la ricezione dei dati dal server. La funzione d'invio, oltre
alla comunicazione con il server, codifica i dati ricevuti in base alle
definizioni presenti all'interno del file \cypher{bro\_fist.h} per mantere
coerenza tra tutte le parti del progetto. Questo permetterà, in caso di
estensione di funzionalità, di non dover cambiare completamente tutto il
codice.

Le porte di questo blocco sono otto in input, sette in output ed una per
l'attivazione esterna. Quelle in input sono una per il \emph{Socket
Descriptor} e sette per le richieste. Quelle in output, come scritto
precedentemente, sono per i dati ricevuti dallo \SPAM{}.

\section[SPAM\_Motor]{SPAM\_Motor: Blocco di codifica dei motori}
\image  {Images/SPAM_Motor}
        {Il blocco SPAM\_Motor}
        {.4}
        {img:SPAMMotor}

Il blocchi \emph{SPAM\_Motor}, similarmente a \emph{SPAM\_Sensor}, servono
per la codifica dei dati indicati dall'utente per l'invio dei parametri per
i motori dello \SPAM{}. A parte i valori da settare per configurare il
blocco questo differisce da quelli per i sensori dalla presenza di una
porta in entrata che avrà settato il parametro da inviare al motore.

La funzione utilizzata da questi blocchi è \linebreak
\codeconst{void bro_motor_enc (scicos_block *, int)}, la quale, oltre
ad assegnare alla porta in output i dati inseriti dall'utente nella
configurazione del blocco, assegna il dato in input alla porta in output.

I valori utilizzabili per indicare come l'utente vuole controllare il
motore scelto sono due:
\begin{description}
    \item[1] per il controllo diretto della potenza del motore, inviando un
        valore da $-100$ a $100$
    \item[2] per il controllo della velocità in \emph{Gradi al Secondo},
        sfruttando il \PID{}, il quale viene eseguito direttamente sullo
        \SPAM{}
\end{description}

\section[MOTOR\_Disp]{MOTOR\_Disp: Blocco d'invio dei parametri per i
motori}
\image  {Images/MOTOR_Disp}
        {Il blocco MOTOR\_Disp}
        {.2}
        {img:MOTORSDisp} 

Il blocco \emph{MOTOR\_Disp} recupera fino a tre parametri da inviare ai
motori dello \SPAM{}.

La funzione collegata a questo blocco è \linebreak \codeconst{void bro_comm_motor_disp
(scicos_block *, int)}, la quale utilizza un'unica funzione, \codeconst{int
bro_motor_set (scicos_block *)}, che contiene il codice per effettuare sia
l'invio che la ricezione di dati dal server, per ridurre al minimo la
possibilità di \emph{deadlocks}.

Come il blocco \emph{SENS\_Disp} anche questo necessita di un'attivazione
periodica esterna per funzionare correttamente.

\section{Problemi dello sviluppo di blocchi per SciCos}
\label{sec:whyitsucks}
Lo sviluppo dei blocchi per \SciCos{} ha messo in evidenza un gran numero
di problemi dati dalla struttura del programma che hanno rallentato il
lavoro non poco.

Faremo ora una lista dei problemi una veloce esposizione dei dettagli.

\subsection{Nessun passaggio di variabili di contesto alle funzioni}
Ogni volta che viene chiamata una funzione da un blocco di \SciCos{}
scritta in \techname{C} a questa vengono passati solo due parametri, ovvero
una struttura definita da \SciCosLab{} ed un \codeconst{int} che funge da
\emph{flag}.

Questa struttura contiene dati relativi solamente al blocco, senza la
possibilità di passare nessun altro tipo di dato, rendendo così
obbligatorio l'utilizzo di variabili globali. Questa è una pessima pratica,
in quanto rende meno comprensibile il codice, oltre che molto più difficile
da mantere.

\'E stato per questo motivo che si è deciso di utilizzare le porte dei
blocchi \SciCos{} per il passaggio del \emph{Socket descriptor}, evitando
così l'utilizzo di variabili glbali.

\subsection{Mancanza di feedback da parte di \SciCos{} in caso di errore}
Uno degli strumenti più importanti per lo sviluppatore è il feedback che un
programma dovrebbe dare un programma quando avviene un errore.

Nel caso di \SciCosLab{} questa funzionalità sembra mancare, in quanto gli
unici errori che sono stati ricevuti durante lo sviluppo dei blocchi
avvisavano solo che c'erano stati ``dei problemi con un elemento della
GUI'' quando era presente un errore nel codice di dichiarazione del blocco
interessato.

Un altro caso si è verificato quando, per un'incogruenza tra
il codice scritto in \techname{C} e la dichiarazione del blocco, veniva
settata una variabile che non era stata allocata da \SciCos{}. L'unico
errore dato da \SciCos{} è stato ``Type $0$ not yet supported by outtb'',
senza fornire alcun informazione di contesto. La correzione di questo
errore ha richiesto moltissime ore di test e di \emph{Retro-Engineer},
finchè non si è isolato l'errore. Questo bug sarebbe stato corretto molto
più rapidamente se l'applicazione avesse fornito più informazioni riguardo
all'errore.

\subsection{Codice sorgente offuscato e senza commenti}
Quasi ad aumentare il peso di questa mancanza il codice sorgente
(rilasciato con licenza Open Source) è stato offuscato ed i commenti sono
stati rimossi.

Per questo la correzione di errori come quello presentato sopra si rende
ancora più complicato. Per fare un esempio, collegandosi a quello sopra
citato, \cypher{outtb} è un array ricavato da un secondo array che contiene
tutti i valori delle porte, il quale a sua volta è contenuto in un terzo
array che contiene tutti i dati della simulazione. La gestione dei dati di
ognuno di questi array viene effettuato utilizzando funzioni non
commentate, rendendo così quasi impossibile la comprensione del
funzionamento del software.

Se il codice sorgente fosse più comprensibile probabilmente la scrittura di
funzioni collegate a blocchi per \SciCos{} risulterebbe meno complicato.

\subsection{Mancanza di una comunità di sviluppatori attiva sul Web}
Gli ultimi due problemi sopra elencati non sarebbero così gravi se fosse
presente una comunità attiva di sviluppatori di \SciCos{} alla quale
rivolgersi in caso di errori o per chiarimenti sul funzionamento del
software.

Basti pensare che sulla rinomata comunità di sviluppatori chiamata
\techname{StackOverflow} non erano presenti domande su
\SciCos{}\footnote{Nè, ovviamente, tantomeno risposte} fino all'arrivo del
redattore di questa tesi, il quale ha provato a presentare un problema
relativo al sistema per quindi non ricevere risposte per settimane e
quindi, dopo aver risolto empiricamente il problema, essersi risposto da
solo\cite{bib:SciCosOnSO}.

\'E stato solo grazie al relatore della tesi che si è potuto contattare uno
degli sviluppatori, grazie al quale si è venuti a conoscenza del fatto che
i blocchi per la comunicazione col server necessitavano di un'attivazione
periodica basata su clock esterno, cosa alla quale non ci si sarebbe
arrivati senza supporto esterno.

\subsection{Stato della documentazione ufficiale}
Lo stesso sviluppatore, però, ha anche consigliato di seguire i manuali
ufficiali per capire meglio il funzionamento del sistema e per avere
un'idea più chiara di come vadano sviluppati i nuovi blocchi per \SciCos{}.

Era stato già utilizzato un manuale ufficiale\cite{bib:SciCosMan1}, fornito
dal relatore, il quale conteneva un capitolo per lo sviluppo di nuovi
blocchi lungo una decina di pagine con solo due esempi, un listato della
struttura \cypher{scicos\_block} e la descrizione di solo una parte dei
parametri utilizzati.

\graffito{La seconda edizione del libro}
Lo sviluppatore, comunque, ha consigliato l'utilizzo della nuova versione
del manuale\cite{bib:SciCosMan2} in quanto la quale conteneva un capitolo
``aggiornato'' completamente dedicato allo sviluppo di blocchi in
\techname{C}.
Scoraggiati dal costo del manuale, 60€, si è deciso di cercare una versione
digitale in internet per accertarsi che l'acquisto avrebbe avuto l'impatto
desiderato sullo svilupppo dei blocchi. Dopo un'attenta lettura queste
aspettative sono state deluse, in quanto l'aggiornamento prevedeva soltanto
la sostituzione del listato della struttura con un paragrafo contenente
meno dettagli su di esso di quanti non fossero presenti nella vecchia
edizione. Oltre a questo i tipi dichiarati per i vari campi della struttura
erano di tipi differenti tra una versione ed un'altra.

\graffito{Il tutorial presente sul sito ufficiale}
Per ovviare al problema si è deciso, in principio, di seguire un tutorial
sulla creazione di blocchi collegati a funzioni scritte in \techname{C}
presente sul sito ufficiale\cite{bib:ScicosCTutorial}. La lettura di questo
tutorial ha reso le cose ancora più confusionarie in quanto scritto da una
persona che, avendo necessità di mettere per iscritto i passaggi fatti per
la creazione dei propri blocchi\footnote{Per avere un esempio in futuro in
caso dovesse crearne altri}, ha deciso di scrivere un tutorial. Questa
scelta, come si può leggere alla pagina $3$ del tutorial, è stata inoltre
influenzata dalla documentazione ufficiale di \SciCos{}, la quale
``manca dolorosamente di esempi completi con descrizioni approfondite dei
passaggi da intraprendere''.

\graffito{La documentazione dei corsi di formazione del 2008}
Infine si è deciso di utilizzare la documentazione ufficiale distribuita
durante i corsi di formazione per sviluppatori di \SciCos{} del
2008\cite{bib:FormationMars}. Al
loro interno erano presenti vari dettagli mai accennati prima nelle altre
documentazioni\footnote{Cosa che fa riflettere sul perché gli autori del
libro non abbiano deciso di aggiungere anche quei dettagli nella loro
pubblicazione}, tra cui il metodo di memorizzazione delle matrici
all'interno degli array di input ed output della struttura
\cypher{scicos\_block}.

Forti di queste nuove conoscenze si è continuato con lo sviluppo dei vari
blocchi, fino a che non si è incappati con un errore dato
dall'interpretazione dei dati.

Secondo due delle documentazioni ufficiali\footnote{Più precisamente il
libro Modeling and Simulation in Scilab/Scicos with ScicosLab 4.4 ed i
documenti dei corsi di formazione} i dati delle porte in input ed output
vengono memorizzati all'interno di array di puntatori a \cypher{void}, il
che permetterebbe la memorizzazione di \codeconst{int} e \codeconst{float}
nello stesso array.

Visto il comportamento del programma\footnote{E ciò che era presente nella
prima versione del manuale ufficiale}, però, si è deciso di andare più a
fondo con la questione ritornando nel codice sorgente. Trovata la
dichiarazione della struttura si è venuti a sapere che il tipo dei dati
delle porte in input ed output è \emph{sempre} \codeconst{float}, a
differenza di quanto specificato nelle ultime versioni della documentazione
ufficiale.

Quindi mi sento di dire che, per quanto siano presenti delle documentazioni
da oltre $300$ pagine, queste sono improntate solamente all'utilizzo del
software e non allo svilupppo di nuove funzionalità, rendendo un compito
già difficile quasi impossibile.

\cleardoublepage

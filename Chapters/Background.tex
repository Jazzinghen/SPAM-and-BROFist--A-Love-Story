\chapter{Background}

In questo capitolo parleremo del background tecnico e culturale che sta
dietro a questo progetto. Ci sar\'a spazio per una breve storia dei \nxt{}, 
delle veloci nozioni riguardanti al \PID{} e del \emph{Bluetooth}.

\section{I MINDSTORMS e gli NXT}
I \nxt{} sono l'ultima incarnazione di una serie di prodotti creati
dalla \emph{Lego}, la quale affonda le sue radici da una serie di sensori
programmabili venduti prima del 1998 dalla \emph{Lego}.

\subsection{Storia dei Lego MINDSTORMS}
La prima versione
della serie si chiamava \RIS{} ed \'e stata rilasciata sul mercato nel
1998, seguita poi nel 2006 dalla versione \emph{MINDSTORMS NXT}. Nel 2009
\'e stata pubblicata l'ultima edizione, chiamata \emph{MINDSTORMS NXT 2.0}.

\graffito{Le origini, prima della \emph{Lego}}
La base hardware e software della prima versione dei \emph{Lego Mindstorms}
\'e derivata direttamente dal lavoro dell'\emph{MIT Media Lab}, i quali
avevano creato dei brick programmabili a scopo didattico. Questi erano stati 
programmati in \emph{Lego Logo} ed avevano un ambiente di sviluppo grafico
creato appositamente dalla \emph{Universit\'a del Colorado} nel 1994 con il
nome di \emph{LEGOSheets}.

\graffito{I vari contenuti dei kits dei \emph{Lego Mindstorms}}
Non tutte le versioni dei \emph{Lego Mindstorms} hanno avuto le stesse
dotazioni in campo di sensori e servo motori. La prima versione chiamata
\RIS{}, conteneva due motori, due sensori di tocco ed un
sensore di luce. La versione \emph{NXT} conteneva, invece, due \emph{Servo}
Motori, un sensore di luce, uno di tocco, uno sonoro ed uno di distanza.
L'ultima versione, infine, ha la stessa dotazione di quella precedente,
solo con due sensori di tocco invece di uno soltanto. 

\graffito{L'uso dei \emph{Lego Mindstorms} in campo didattico}
I \emph{Lego Mindstorms} sono derivati, come accennato sopra, da un applicazione 
di tipo didattico dell'\emph{MIT Media Lab}. Per questo la Lego ha deciso
di continuare a seguire questa idea proponendo alle scuole dei set
didattici chiamati \emph{Lego Mindstorms for Schools}. Questi set,
conosciuti anche come ``Challenge set'', differiscono da quelli in
commercio normalmente per un sensore di luce extra, alcuni ingranaggi in
pi\'u ed un software per la programmazione dei brick differente.

Questo software, chiamato ``ROBOLAB'', \'e basato sul software ``LabVIEW''
della \emph{National Instruments}, un ambiente di sviluppo per il
loro linguaggio di programmazione visivo per l'automatizzazione
industriale. Un altra scleta per la programmazione dei bricks \'e
l'utilizzo di firmware o linguaggi di programmazione di terze parti, tra i
quali anche linguaggi conosciuti come il \emph{C} o il \emph{JAVA}, i quali
vengono usati dai professionisti in campo di programmazione embedded.

\subsubsection{L'origine del nome \emph{Mindstorms}}
I \emph{Lego Mindstorms} devono il loro nome al titolo del libro
``Mindstorms: Children, Computers, and Powerful Ideas'', scritto da \emph{Seymour
Papert}, uno dei pionieri dell'Intelligenza Artificiale ed inventore del
\emph{LOGO}. In questo libro Papert proponeva un ambiente di studio basato
sui computer chiamato \emph{il Micromondo}.

Questo Micromondo doveva completare il metodo di costruzione della
conoscenza dei bambini, conosciuto come \emph{Approccio Costruttivista
all'insegnamento ed al sapere}.

Siccome i \emph{Lego Mindstorms} hanno alla base proprio questa idea
perci\'o la \emph{Lego} ha deciso di utilizzare questo nome.

\section{Lego MINDSTORMS - RCX}
La prima versione dei \emph{Lego Mindstorms} aveva come base il brick
conosciuto come \emph{RCX}.
\graffito{Specifiche tecniche dell'RCX}Contiene un microcontroller 
\emph{Renesas H8/300} come CPU ed una memoria RAM di 32KB. L'unica
interfaccia di comunicazione con il PC o gli altri RCX \'e una porta
\emph{InfraRed} (IR). Questo rende impossibile l'upload di programmi
sull'RCX con metodi differenti da un'interfaccia proprietaria IR per la
quale esistono drivers solo per Windows e MAC. \'E possibile caricare
programmi anche in ambiente \emph{*NIX}, utilizzando per\'o metodi
differenti, quali il compilatore per \emph{Not Quite C}.

Oltre alla porta IR sono presenti tre porte per i sensori, tre per i
motori ed un LCD sul quale si poteva leggere la carica della batteria, lo
stato di motori e sensori, il programma in esecuzione oppure quale
programma era in esecuzione al momento.

I programmi, sull'RCX, vengono mantenuti sulla memoria RAM, quindi ogni
volta che veniva spento il brick questi vengono cancellati ed vanno
ricaricati sul brick. Questo problema, nella versione 1.0 degli RCX, \'e
risolto grazie alla presenza di un jack per la connessione ad un
caricatore.


\cleardoublepage
